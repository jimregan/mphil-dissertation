\chapter{Citizen's Information Corpus}
\label{app:citinfcorpus}

Citizen's Information, available at \href{http://www.citizensinformation.ie}{http://www.citizensinformation.ie}, 
is an Irish government website run by the Citizens Information Board, a national agency responsible for
providing information about government services. 

As a source of parallel text, it has a number of desirable qualities: first and foremost, a large amount of its
content has been translated from English to Irish, while a smaller number of documents are additionally 
available in French, Polish, and Romanian.

Where translations are available, they are highlighted in a box on the right hand side of the page; although
this in itself makes finding pairs of parallel texts quite simple, the address design of the site makes it
even more simple: each document is prefixed first by its two-character ISO 639-1 code, followed by categories
and subcategories the document fits into, followed finally by the page title. Translated documents retain the
English address, changing only the ISO 639-1 language prefix.

Where a translation has been removed -- as happens when the English original has been updated, but the translation
lags, and would otherwise provide inaccurate information -- the page redirects to the English version. This in
itself would normally make it necessary to run language identification software over the non-English documents, to
verify that the document is in the language that the ISO 639-1 prefix of the address would suggest, but the site
generator makes this unnecessary: the \texttt{<html>} tag, the root level element of all HTML documents, is
generated with a \texttt{lang} attribute that contains the correct ISO 639-1 language code, thus making language 
identification unnecessary.

Boilerplate removal is an important step in producing any kind of corpus from web-based texts. One reason why,
in general, is that the inclusion of frequently repeated text items, such as those appearing in site 
navigation menus, can lead to misleading statistics. For parallel text in particular, the inclusion of a
language-choosing element can itself lead to misleading translations: the spot offering ``B\'earla'' in the
Irish version of a website would quite naturally offer ``Irish'' in the same location of the English version
of the page, leading to an incorrect pairing of ``B\'earla'' and ``Irish''.

The high quality page generation of the Citizen's Information site again provides assistance. On each generated
page, the content of the page is separated clearly from the boilerplate with cues in the HTML markup:

\begin{verbatim}
<!-- start of Document -->
<a name="startcontent"></a>
\end{verbatim}

One of the most attractive qualities of the Citizen's Information website is that it is one of the
first government-run websites to fully embrace the recommendations on the Re-use of Public Sector Information,
with all content of the site available under the recommended Open Content licence, Creative Commons CC-BY 4.0.
The terms of this licence allow content that is covered by it to be re-used, modified, and/or redistributed,
as long as an appropriate acknowledgement of the source of the information is provided. This permissive
licence means that any linguistic resources resulting from the website can be used at will, without relying
on research exemptions, and can therefore be used by commercial entities as well as by other researchers.

Finally, that the content of the Citizens Information site is available as HTML is an advantage. Revenue.ie,
the Irish Tax and Customs website, is also available under an open licence, with a large amount of parallel
text in English and Irish, but the majority of its documents are available as PDF. HTML, as a format designed
for on-screen display, is much easier to align at a document level: for example, paragraphs are marked
semantically, with the decision of how they ought to be rendered left to the web browser -- the site designer
may set a number of properties using style sheets, etc., to control how they are presented, but these
properties can be ignored, which is quite often desirable when the reader is using technological solutions
for accessibility, such as screen readers for the blind.

PDF, on the other hand, is designed for print layout. Although PDF can additionally contain the plain text
that is visible within the document, it is not always the case that it is present; even when present, the
PDF creation software may have used its own internal encoding system for rendering characters, such as vowels
with fadas; also, if the document was not ``digital native'', but created through scanning, the text may have
been added using Optical Character Recognition (\GLS{OCR}), which may introduce errors of its own: a common
example is ``arid'' in place of ``and'', due to the visual similarity of \texttt{ri} and \texttt{n}.

\section{Conversion}

Because of the relatively small amount of processing required, due to the nature of the Citizen's 
Information website, much of the usual processing involved in preparing a bilingual corpus from the web
could be bypassed: documents were paired based on their address, language content checked based on that
reported in the \texttt{<html>} tag, and a simple script (included in the supplemental data repository, and
is also reproduced in Appendix~\ref{app:bpremoval}.