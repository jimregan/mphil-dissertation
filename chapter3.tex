% chap3.tex (Definitions and Theorem)

\chapter{Development}
\label{chap:devel}
%\epigraphfontsize{\small\itshape}
\epigraph{\textit{\small{I promised my right hemisphere it was never going to have to work full time}}}{``Oliwka N.''}

\section{Rule-Based Machine Translation}
\label{sect:ch3rbmt}

In this section, the development of the RBMT system will be described, covering the resources used to create the system, 
tools created to convert those resources and to more efficiently employ them, as well as outlining difficulties encountered.

The creation of the RBMT system was part of an internship, which came with two primary
requirements: the conversion of the bilingual data (see~\ref{ssect:devrbmtbil}) and the
creation of a set of transfer rules.

To this initial set of requirements were added the additional tasks of converting 
IrishFST~\citep{elain09} for the analysis and generation of Irish, and the creation
of a tool to allow the specification of rules in a manner that either allowed or
approximated recursion in the rules.

\subsection{Bilingual data}
\label{ssect:devrbmtbil}

The bilingual data used for the RBMT system is derived from data provided
by The Department of Culture, Heritage and the Gaeltacht, consisting of
English-Irish Dictionary (EID)~\citep{debhaldraithe1959english} and Focl{\'o}ir 
Gaeilge-B{\'e}arla (FGB)~\citep{donaill1977focloir}, from which permission was
granted to create an open source lexicon to contribute to the Apertium
project.

\subsubsection{Dictionary cleaning}

The only source of information available on the XML editions of the dictionaries
is from a blog post\footnote{\href{https://multikulti.wordpress.com/2014/01/04/how-to-retro-digitize-a-dictionary/}{https://multikulti.wordpress.com/2014/01/04/how-to-retro-digitize-a-dictionary/}} 
that describes the process of creating the XML from the source data.

The blog post mentions that EID had been scanned, OCRed, proof-read and marked
up into XML, but that the XML structure had been lost during the proofreading
process, and that attempts were made to infer the structure based primarily
on the formatting that remained. FGB, on the other hand, was generated from
the original typesetting files.

%FIXME: Kevin did the OCR, mentioned that 30\% of the source words are not real words, etc.

An initial extraction of the simplest (and most common) types of entry layout,
based on a na\:ive regular expression-based extraction,
had a quite high yield: 13076 entries, but missed out on some quite
common words: for example, the entry for ``girl'' contains two subsenses, and 
a number of empty sub-subsense entries (which in the original are explained by
example, rather than by direct translation). The simplistic parser was unable 
to account for these structures, and I was tasked with parsing the XML in a more
robust manner.

This loss of structure has proven to a quite difficult obstacle, and the first two
attempts at parsing the XML directly lead to failure. The first attempt used
a Java-based parser that attempted to read the sub-elements of each entry in a
linear fashion, but attempting to compensate for some of the idiosyncracies in
the XML was ultimately unsuccessful. A second attempt used Scala, and leveraged
Scala's pattern matching facilities to emulate the regular expression-based
extraction for simpler entries, with a fall-back to the serial approach used
in the Java parser for complicated entries that did not match those simpler
entries\footnote{As Scala runs on the JVM, Java-based classes are essentially 
native to it.}. An initial attempt was made to make corrections to the XML structure
while parsing, but this swiftly became over-complicated and unmanageable.

While attempting to correct the structure of the XML, I came across some 
erroneous components of the entries that were also present in the online
version of the dictionaries; I therefore decided to revisit my strategy for
parsing the XML, by making it a layered process.

The first layer addresses mistakes in the XML in a simple perl script that
uses regular expression-based transformations to the text before it is parsed.
As well as simplifying the XML input, separating this stage of the parsing
has the added advantage of being packaged in a way that can be easily passed 
back to the maintainers of the online version.

The scripts handle a number of different issues in the source, from errors to minor 
inconsistencies. The difference in the structure of both dictionaries is perhaps 
reflected in the number of transformations performed by each: 193 transformations 
in EID, compared to 76 in FGB.

Taking into account that EID was scanned, OCRed, and proofread, there are surprisingly 
few OCR errors: at the time of writing, I have found only one such error\footnote{There 
is also ``clo dubh'' given as translation for ``black-letter'', instead of ``cl\'o dubh''; 
and ``duct aeir'' rather than ``ducht aeir'', but as I have not verified these entries 
against the printed edition I cannot tell if these are transcription errors.}, where 
``hurling'' was translated as ``lom\'ana\'iocht'' instead of ``Iom\'ana\'iocht'': that 
is, with a lowercase ``L'' in place of an uppercase ``i''. 

Most of the translations are aimed at consistency: for example, where a translation has
 been mislabeled as a categorical label, or in merging separate translation elements that 
would otherwise lead to incorrect translation entries being generated (``etc.'' is a 
frequent offender).

The current version of the parser operates on the transformed XML. As well as having 
more consistent input than the previous attempts, it no longer attempts to parse the 
entries in a purely linear fashion: for example, the element \texttt{$<$b$>$} in the 
XML edition of FGB serves two purposes: when following $<$g$>$ it contains a word form, 
or information from which to construct one.
In other contexts, it can contain a subsense number, preceding a 
translation of that other sense of the word.

In both dictionaries, this re-use of elements is a frequent source of errors: one 
particular example is of the \texttt{$<$r$>$} element in FGB. In a ``see also'' context,
it signifies that a Roman numeral is part of the entry reference; when contained by a
\texttt{$<$trans$>$} element, it contains the translation(s); and elsewhere it is used
to contain usage information in parentheses. One frequently occurring problem in FGB is that some
pieces that ought to have been marked as translations have not been, presumably because 
the translation began with a parenthetical piece (most commonly ``(act of)'').

The current parser, upon reading a $<$g$>$ element, checks if the next is $<$b$>$, and if so, 
it is consumed by the $<$g$>$ element. As well as simplifying the parsing process, this ensures 
that the grammatical forms are kept with the information about those forms.

\subsubsection{Entry replacement}

As Irish has no single citation form for verbs, it is common to either use the first person singular 
present tense form (``t\'aim'', ``scr\'iobhaim'') or the second person imperative singular (``b\'i'', 
``scr\'iobh'') as the citation form.

It was clear from the outset that the translations of verbs in EID would need to be replaced, primarily 
for consistency, as it uses the first person form of the verb, while both FGB and the morphological 
analyser use the imperative. These replacements were quite simple to generate, using the analyser, but 
the generation process uncovered a more pressing reason for replacing verb translations: the ``compression'' 
of shared elements in verb phrases. For example, the entry ``yowl\textsuperscript{2}'' contains the translation 
``Ligeann, casann, uaill'', where the common element of the verb phrases ``ligeann uaill'' and ``casann uaill'' 
is shared among them, for conciseness.

This changed the replacement process into a two-stage process: first, the whole translation text is 
compared against a list of replacements for multiwords of this kind, which additionally contains pattern 
rule information, to generate a single entry from the multiword; if the entry is missing from this list, 
it is split by comma and semi-colon, and each individual word is checked against the original verb form 
mapping list. If the word is absent from this list, it is marked to be ignored: an output entry is created, 
but flagged so that it will not be used for translation -- words that cannot be generated by the analyser 
cannot fully be utilised by the translator, as they cannot be inflected.

Later private correspondence with Kevin Scannell\footnote{A noted developer of language technologies for 
Irish and other Celtic and minority languages} revealed not only that he had OCRed EID, and created the 
XML structure that had been lost, but his later work with a Manx translation of it had caused him to delve 
a little deeper into the history of EID, which had, in turn, been based on an earlier French-English 
dictionary. He warned that 30\% of the English headwords make no appearance in a corpus of contemporary English.

Taking this warning on board, I checked first the list of words that contain a hyphen in the name, some of 
which (such as ``black-letter'' and ``teen-ager'') had a distinctly archaic appearance. This lead to the 
addition of a similar replacement mechanism for headwords; but, as FGB shared at least some history with EID, 
I checked the translations of some of the hyphenated words against the list of words to be replaced in EID, 
and found the same need for replacement there. Fortunately, the high degree of overlap meant that the same 
replacement list could perform double duty. The list itself was, for the most part, automatically generated, 
by generating both a single word and separated words from the hyphenated form, and checking these against 
external lexical sources for English.

Extending my inspection of translations beyond verbs, I found that a similar need for 
replacement existed in other categories. For the most part, this needed nothing more 
than the addition of the lists of multiword generation patterns that were already 
required to treat the multiwords as a single unit, but where a whole translation needed 
to be replaced, this required something extra in EID, because of the presence of 
grammatical elements to mark gender.

The recognition that these replacement mechanisms formed a basis for addressing language 
change, by replacing the contents with something more relevant to the living languages, 
lead to a further replacement stage, to precede the others: at the time the entry title 
and its primary sense key are read, another list is consulted, from which a replacement 
for the entire entry might be taken. 
For example, figure~\ref{fig:commission2} contains the definition of the verb 
``commission'' from EID; in modern use, all of these senses have been replaced 
by the verb ``coimisi\'unaigh'', which does not appear in either dictionary. 
This entry replacement mechanism allows the whole entry to be replaced by this 
single translation.

This mechanism allows the insertion of a different headword, where appropriate: 
``ruaille\textsuperscript{2}'' in FGB exists to stand for ``ruaille buaille''. 
As this is treated as a phrase, the XML entry contains no useful information. 
With this mechanism, ``ruaille buaille'' can be inserted in place of ``ruaille''.

\begin{figure}
\textbf{commission}\textsuperscript{2}, \textit{v.tr.} 1 \textbf{a} Tugaim údarás do; údaraím. \textbf{b} Ainmním (duine) ina oifigeach; tugaim coimisiún do (dhuine). \textbf{c} Ordaím, iarraim (leabhar a scríobh, etc.). 2 \textit{Nau:} Armálaim, feistím (long).
\label{fig:commission2}
\caption{Definition of commission\textsuperscript{2} from EID.}
\end{figure}

\subsubsection{Current status}

The latest extraction produced 88243 distinct entries. Although there is significant 
overlap with the previous 13076 entries, it is not a complete overlap: the differences 
need to be inspected, to ensure that the differences are not due to regressions 
introduced in the new code.

\subsection{IrishFST conversion}

Translators created using Apertium use, for the most part, a common set of ``tags'' to
describe linguistic features: \texttt{n} for nouns, \texttt{vblex} for (lexical) verbs, etc.
This is in part due to the fact that the same small group of people have been involved
in the development of the majority of language pairs, but primarily to simplify the
transfer process: the same attributes will, by default, be copied from source to target,
so features such as \texttt{sg} (singular) and \texttt{pl} (plural) do not need to
have any specific action added in the default case. To take advantage of this, the 
tagset used in IrishFST needed to be adapted.

The first pass at performing this adaptation went quite easily, aside from a few
corner cases where a pair of tags in one corresponded with a single tag in the other.
More difficult, however, was the conversion of the lexicon to Apertium's conventions.

Apertium uses a simple concatenative model of inflectional paradigms, where the common
substring of the word has a paradigm reference added to it, from which the various forms
of the word are expanded. IrishFST uses a more complex system where the continuation class
of a word is added in its entry, from which the individual forms are computed, based on
the word's stem.

An initial set of conversions was performed using an Apertium tool for generating paradigms
for an expanded list of forms; this tool, however, can only accommodate suffix inflection.

One other difference between the systems is that Apertium's lexical processing unit, thanks
to the use of ``tokenize-as-you-analyse'', is able to process multiword units as words-with-spaces;
it also has a facility for breaking apart contracted words; to simplify the transfer of Irish
conjugated prepositions, such as ``liom'' (with me), I opted to convert these prepositions to
follow Apertium's conventions, which allows conjugated prepositions to be treated the same
as unconjugated prepositions, deferring their combination to the orthographic post-processing
module. This also allows a more satisfactory treatment of genuine contractions in IrishFST,
such as ``s\'eard'', a contraction of ``is \'e an rud'' (\textit{that's the thing}), which
can be split into its components, rather than, as in IrishFST, being tagged with the features
of each, but with only the lemma of the first.

The resulting converter, in comparison with the original conversion script, is quite complex.
It performs the same paradigm generation as the existing Apertium tool, but takes into
account Irish prefix mutations. This facility also allowed the pinpointing of several errors
in IrishFST, where a number of prefixed forms were missing tags to represent the mutation.

\subsection{Engine modifications}

The needs of Irish required some changes to Apertium's lexical processing tools. Unmodified,
Apertium's morphological generation component was unable to generate the correct form when
a word beginning with a vowel was prefixed with ``n'' or ``t''; attempting to add separate
entries for both upper and lower case word forms lead to two outputs: one correct, as well
as the original incorrect form. I modified Apertium's lexical processor to add a mode that
allows forms specified in this way to have a unique, correctly generated form.

In addition to this, because of the nature of much of the publicly available bilingual data
for Irish (see \ref{ssect:smtdata}), I extended the lexical processor to allow it to be
provided with a list of characters to be ignored in processing, intended to discard some
of the non-printing characters in Unicode, such as soft hyphen -- a character which is
normally invisible, used to provide a place to hyphenate a word, to split it across lines in
justified text.

Although not strictly necessary, the modifications that made ignoring specified characters
possible were similar to the changes that would be necessary for diacritic restoration; as
this was now a relatively simple change, I added it, as quite a lot of publicly available
Irish text is written without the use of fada (the acute accent).

All of these modifications have been accepted into the public version control system of
Apertium, and will be included in the next public release of the software.

\subsection{Rule conversion}

To meet the secondary requirement of a recursive, or quasi-recursive, means of specifying
transfer rules, I used \texttt{apertium-transfer-tools}~\citep{sanchez06a} as a model (and,
to a lesser extent, its successor, ruLearn~\citep{sanchez-cartagena16b}): this set of tools
are designed to create Apertium rules from a parallel corpus, using techniques from Statistical
Machine Translation.

The basic format of such rules -- a set of words in both language, and their alignments -- seemed
vastly preferable to the extremely verbose XML rules used by Apertium. The first limitation of
this toolset is that it only generates rules appropriate for a single level of transfer, while
it was quite apparent that for Irish-English, chunking would be necessary. ruLearn attempts
to augment the original transfer tools package with a means of generating chunks, but it can
only do so when there is an Apertium chunker already available, which is not the case for Irish.

Adding to this, there are some significant mismatches between chunks in English, and those in
Irish. One example is verb phrases in the continuous aspect: ``be going'' is (arguably) a single
chunk in English, but the Irish ``b\'i ag dul'' must be two, to allow the subject to be placed
after the finite verb ``b\'i'', and before the verbal noun phrase ``ag dul''. 

Another complication is that, in contrast with a language such as Spanish, where the base case of
most Apertium rules is typically quite regular, most chunking rules between English and Irish
will have more than one regular correspondence: in phrases containing a determiner, if it is
the English indefinite article, it must be omitted; if it is ``this'' or ``that'', the rest of
the noun phrase will have the definite article ``an'' as its leftmost component, and ``seo'' or
``sin'' as the rightmost; for a verb phrase, the preferred translation could be a verb, or, in
the case of a number of frequent verbs, the preferred Irish equivalent could be a copula phrase:
``like'' becomes ``is maith le'', ``prefer'' becomes ``is fearr le'', etc.

Apertium's transfer system allows common operations to be placed in macros, which can be shared
among rules, simplifying them greatly. One common macro operation necessary for English to Irish
is to add the type of a determiner to the chunk's tags: Irish genitive constructs have a definiteness
restriction, much like that of the Anglo-Saxon genitive in English: only one component may be definite;
at the level of an individual chunk, it cannot be known if the chunk is part of a genitive (or
prepositional) phrase, so the chunk is tagged for interchunk, where it can be known, and during
reordering, the tag changed as a signal to the postchunk processor to remove the article before
output. Similarly, where both sides of a phrase that \textit{seems} like a genitive use some
other definite determiner -- ``my photos of my holiday'' -- this chunk annotation is used to
replace the preposition with ``de'', instead of using a genitive construct.

Rule conversion, rather than replacement of the transfer mechanism, was selected because developing
a replacement transfer component was not feasible within the timeframe of the internship, and because
the prior experience of Apertium has shown that fixed-length patterns can be sufficient.

I became overly focused on finishing the rule converter. The first approach I took to recursive
expansion was overly complicated, trying to expand both sides of the rule at the same time, while
attempting to keep the positional indices used by the macros updated. In the week before the
deadline, I thought of a simpler approach, where each token contained both source and target sides
of the rule, and words inserted into the target side were appended; after expansion, these extra
parts are discarded from the source side, while on the target side the tokens are reordered,
according to their position.

While this part of the rule converter works, the more important part does not: as each multi-part
rule contains multiple versions of the same tokens to be expanded, which must be kept together
to maintain the rule as a whole, the expansion method for single-part rules simply cannot work.
The realisation came too late to recover from, and while work is ongoing to expand the rules
by hand, there is simply too much left to do to even attempt to run the system.

\section{Statistical Machine Translation}
\label{sect:ch3smt}

In this section, the details of training SMT systems will be described, including a description of the 
corpora employed. A particular point of interest in SMT are the efforts to incorporate linguistic information into the SMT 
process, such as factored models and hierarchical models, which seemed closest, at least in spirit, to
the knowledge-driven approach.

The purpose of investigating SMT was originally intended as a means of comparison
with the rule-based system, of acquiring data with which to expand it, and to investigate a possible 
method of hybridisation, where the rule-based component was to be used as a fall-back, almost in the
opposite manner to \citet{sanchezmartinez09d}: the input was to be segmented into light-weight, marker-based
chunks~\citep[e.g.,][]{gough2004robust}, but instead of inserting statistically-derived phrases into the output of the rule-based system, 
as in that work, each chunk would be checked to see if it was present in the statistical phrase table; 
if not, it would be passed to the rule-based system, as a fall-back, using the XML mechanism provided
by Moses to insert translation candidates from external sources.

This, however, was not to be: the reality of the data most commonly available -- and, as such, most likely
to be used in the training of SMT systems, generally -- was that it suffered quite severely from issues
of data quality.

\subsection{Data}
\label{ssect:smtdata}

\citet{ARCAN16.9} list a number of available sources of parallel text for English Irish MT (reproduced in figure~\ref{fig:smtcorpora}); however, I have concentrated
on using only parallel text available from the OPUS project~\citep{TIEDEMANN12.463}. Initially, this was simply as a test after
installing Moses; however, as OPUS is perhaps the only large scale effort to make available parallel texts in multiple languages,
and is both widely advertised -- for example, OpenNMT's FAQ\footnote{\href{http://opennmt.net/FAQ/}{http://opennmt.net/FAQ/}}
lists only OPUS as a source of parallel text -- and widely used, it seemed the most likely source of data
for anyone who would attempt to build an SMT system for English and Irish.

Of the subcorpora available as part of OPUS that contain text in Irish, two in particular stood out as the potential
sources of unusual translations from the baseline model: the EU Bookshop Corpus~\citep{SkadinsEA:LREC14} and the
KDE4 corpus~\citep[s. 2.3]{Tiedemann:RANLP5}.

In addition to the existing parallel data, I created a corpus by scraping the Citizen's Information website.
As I have not used it in my SMT experiments, I have not discussed it in this section, but as it may be a
useful resource, I have provided a brief description of its construction in appendix~\ref{app:citinfcorpus}.

\subsubsection{EU Bookshop Corpus}

The OPUS EU Bookshop Corpus was created by crawling PDF documents from the EU Bookshop website. 
As many of these documents are relatively old, a large amount of the text available in these PDF documents has been 
inserted using Optical Character Recognition (\GLS{OCR}); an additional complication is that, due to the nature of documents
produced by the EU, these documents can quite legitimately contain sections or phrases in other EU languages. Because of
this, the OCR system in use for these documents has been trained not on a single language, as is conventional, but on all
EU languages. In a conventional, single language OCR system, there are certain issues in accuracy due to visual similarity
of character combinations within the language: \texttt{m} and \texttt{in} can look similar, and can therefore be confused.

This is a widely known problem, and tools exist to tackle many of these issues, for single languages. The OCR text produced 
by the EU Bookshop, however, has the additional complication of multiple scripts, and even within a single script, there
is a confusion of accented letters: as well as, for example, the visual similarity of many letters of the Greek or Cyrillic
alphabets -- Greek Ρ for Latin P, etc. -- which existing tools do not take into consideration, there is also a confusion
of accented characters, with the acute accent (Irish fada) being replaced with umlaut, macron, and grave accents.

As well as the issue of OCR, the EU Bookshop Corpus contains a number of lines where text boundaries have been incorrectly 
detected in the process of conversion from PDF, with an automatic correction method employed to remove spaces from the
words~\citep[p. 1851]{SkadinsEA:LREC14}; this, however, does not take OCR errors into account, as in figure~\ref{fig:ocreubkshp}.

\begin{figure}
\begin{verbatim}
'Article l5l
If, within six weel < s of its being convened,the Conciliation 
Committee approves of th9 v9te1 aiäint text
\end{verbatim}
\label{fig:ocreubkshp}
\caption{Examples of OCR errors from the EU Bookshop corpus.}
\end{figure}

As the correction method employed does not account for such errors, it cannot
correct, e.g., `weel < s' to `weeks', nor can it correct the word boundaries of
the non-existant word. This issue of spacing is possibly the result
of wider spaces between words in justified text, which would also explain the existence of the opposite problem: missing
spaces between words~\citep[p. 1853]{SkadinsEA:LREC14}.

In testing a sample set of 200 sentences of English-Latvian~\citep[pp. 1852--1853]{SkadinsEA:LREC14}, it was found that
only 59.8\% of the sentences were considered ``good''. A set of automatic corrections were applied to that language
pair, with the resulting, much smaller, corpus achieving 85\% of sentences being considered ``good'' by human raters.

\subsubsection{KDE4 Corpus}

Unlike the EU Bookshop corpus, which was known to have quality issues due to its source data, the KDE4 corpus, which
originates in the localisation files used for programs in the KDE4 desktop environment, comes with no quality warnings.
This is not, however, to say that it does not contain quality issues, many of which seem to have been introduced by the 
conversion process. One example, visible in figure~\ref{fig:kdeenc}, is the mistaken recoding of texts from the 8-bit 
Latin1 encoding, to UTF-8. Also in that example, the segments have misaligned, possibly due to incorrect insertion on
the English side, or incorrect deletion on the Irish.

\begin{figure}
\begin{verbatim}
Shrink Icons	& Ollmhór
& Default Size	& An- mhór
& Huge	& Mór
& Very Large	& Measartha
& Large	& Beag
& Medium	An- bheag
& Small	Cumraigh an Cúlra...
\end{verbatim}
\label{fig:kdeenc}
\caption{Align of incorrectly converted and misaligned segments from KDE4.}
\end{figure}


Secondly, instructions to translators have been included, as shown in figure~\ref{fig:kdetrans}.

\begin{figure}
\begin{verbatim}
Your names	Kevin ScannellEMAIL OF TRANSLATORS
Your emails	kscanne at gmail dot com
January	Month
\end{verbatim}
\label{fig:kdetrans}
\caption{Instructions to translators misidentified as translations, from KDE4.}
\end{figure}

Finally, there seem to have been misaligned files in the corpus, as in figure~\ref{fig:kdemisalign}:
the translations on the English side seem to refer to a different program to those on the Irish.

\begin{figure}
\begin{verbatim}
Search Box	Sonraí Bibleagrafaíocha (BibTeX) Name
Newspaper activity	Sonraí Bibleagrafaíocha (BibTeXML) Name
\end{verbatim}
\label{fig:kdemisalign}
\caption{Misaligned segments from KDE4.}
\end{figure}

\subsubsection{Data cleaning}

\citet{GJ2008} present a method of automatically increasing the coverage of Translation Memory systems,
by allowing the specification of simple rules, using regular expressions, to automatically translate
items such as dates, times, etc. During an earlier phase of my internship, I adapted the Duckling
parser\footnote{\href{https://github.com/wit-ai/duckling\_old}{https://github.com/wit-ai/duckling\_old}} 
for Information Understanding to Irish: it parses similar items, converting them to a 
language-independent representation, which seemed an ideal way of implementing a similar system of
extending Translation Memory: the choice of programming language -- Clojure, a language that uses Java's JVM -- seemed a good fit with
existing open source Translation Memory systems, all of which (that I am aware of) are written in Java.
Although Duckling is able to parse such expressions, it is unable to generate them, and adding this
facility did not seem an appropriate use of time; additionally, Facebook (the developers of Duckling)
later rewrote it in a different, non-JVM language\footnote{Facebook converted my code, so the new version
retains support for Irish: \href{https://github.com/facebookincubator/duckling}{https://github.com/facebookincubator/duckling}}.

Faced with the various errors in the EU Bookshop and KDE4 corpora, I developed a proof-of-concept program
named Duckegg\footnote{Named both reflect the influence of Duckling, and as a nod to a former teacher, who
would refer to a zero in a test as ``a big fat duck egg'', in reference to the fact that, due to the relative 
haste in which it was written, the software will not win any design awards.} to perform a number of automatic
corrections.

Duckling inspired it in two ways: first, Duckling's pattern matching can use regular expressions, or can build
more complex patterns based on existing patterns; second, Duckling's patterns aim to be robust to informal writing.

Most of the proof-of-concept corrections implemented in Duckegg are specific to the gaois.ie corpus (referred
to as ``Irish legislation'' in figure~\ref{fig:smtcorpora}): one particular check is for Statutory Instrument
numbers, which frequently occur in the corpus. Although the English side of the corpus is quite clean\footnote{The
text is publicly available on \href{http://www.irishstatutebook.ie/}{http://www.irishstatutebook.ie/}}, the
Irish text shows a number of OCR errors, so the relevant check compares the Irish with the English, using
a fuzzy match that takes into account common OCR errors in numbers.

Another tool provides a variation of the common ``find and replace'' operation, but adding translation context:
the replacement will be made if and only if the search patterns for both source and target languages match.
For example, ``dta'' could be a corruption of the Irish ``d\'ata'', or it could be an abbreviation that has been
lowercased. If the English side contains the word ``date'', the replacement can be made with more certainty.

Another tool targets a specific set of errors due to a mismatch in tokenisation: a number of the English sentences
end with a phrase like ``(No. 42)'', whereas, due to poor or missing tokenisation on the Irish side, their 
translations end with ``(Uimh.'', without the number. The tool finds such sentences and appends the number and
right bracket from the English side to the Irish.

Finally, another tool checks if either source or translation is missing, or is identical to the other side, to target
a large number of sentence pairs where the English was placed on both sides of the translation. Although, due to
the nature of the corpus, there are a number of instances where the sentences contained amounts in euros related to fines, and are therefore
legitimately identical on both sides, omitting these number-only translations is not likely to have much of a negative
impact on an SMT system trained on such data; the same cannot be said about including hundreds of sentences of English
in place of Irish.

One other check that was attempted, but ultimately discarded, was influenced by the presence of the names of
translators in the KDE4 corpus. I attempted to train a statistical Named Entity Recognition (NER) system, with the
intent of checking both sides of the corpus to ensure both sides contained equivalent entities, and to print
a warning if not. Due to the nature of the training data, the NER system is quite successful at recognising
Irish names (i.e., containing a patronymic particle), but quite poor otherwise. This meant that the matching
of the Irish NER model was much less capable than the pretrained English model, so the experiment was
ultimately abandoned as it was unable to work as intended.


\begin{figure*}
\begin{center}
\begin{tabular}{l|r|r|r}
\hline
Corpus & \# lines & \# English words & \# Irish words \\
\hline
DGT & 36,275 & 864,373 & 950,500 \\
EU Bookshop & 121,042 & 2,606,607 & 2,704,091 \\
EU constitution & 6,267 & 125,553 & 126,355 \\
Focale & 213,683 & 414,730 & 440,228 \\
GNOME & 75,051 & 288,916 & 297,882 \\
Irish legislation & 132,314 & 2,691,928 & 2,792,595 \\
KDE4 & 110,138 & 439,273 & 523,614 \\
News-2007 (English) & 3,782,548 & 90,490,396 & / \\
Ubuntu & 191 & 1,038 & 1,103 \\
Wikipedia Titles & 17,421 & 35,165 & 36,760 \\
Irish sent. bank & 3,895 & 31,655 & 32,800 \\
Food and Beverages & 339 & 696 & 712 \\
Wikipedia (Irish) & 246,290 & / & 4,047,229 \\
Textbooks & 373,401 & 5,929,635 & 6,568,295 \\
Apertium & 720 & 804 & 791 \\
\hline
total (parallel) & 1,096,117 & 13,573,234 & 14,684,356 \\
\hline
\end{tabular}
\end{center}
\caption{Summary of available corpora for English and Irish, reproduced from \protect\citet[p. 568]{ARCAN16.9}}
\label{fig:smtcorpora}
\end{figure*}

%OpenNLP namefinder - person
%       TOTAL: precision:   91.79%;  recall:   86.82%; F1:   89.23%.
%      person: precision:   91.79%;  recall:   86.82%; F1:   89.23%. [target: 14631; tp: 12702; fp: 1136]


%\chapter{Citizen's Information Corpus}
\label{app:citinfcorpus}

Citizen's Information, available at \href{http://www.citizensinformation.ie}{http://www.citizensinformation.ie}, 
is an Irish government website run by the Citizens Information Board, a national agency responsible for
providing information about government services. 

As a source of parallel text, it has a number of desirable qualities: first and foremost, a large amount of its
content has been translated from English to Irish, while a smaller number of documents are additionally 
available in French, Polish, and Romanian.

Where translations are available, they are highlighted in a box on the right hand side of the page; although
this in itself makes finding pairs of parallel texts quite simple, the address design of the site makes it
even more simple: each document is prefixed first by its two-character ISO 639-1 code, followed by categories
and subcategories the document fits into, followed finally by the page title. Translated documents retain the
English address, changing only the ISO 639-1 language prefix.

Where a translation has been removed -- as happens when the English original has been updated, but the translation
lags, and would otherwise provide inaccurate information -- the page redirects to the English version. This in
itself would normally make it necessary to run language identification software over the non-English documents, to
verify that the document is in the language that the ISO 639-1 prefix of the address would suggest, but the site
generator makes this unnecessary: the \texttt{<html>} tag, the root level element of all HTML documents, is
generated with a \texttt{lang} attribute that contains the correct ISO 639-1 language code, thus making language 
identification unnecessary.

Boilerplate removal is an important step in producing any kind of corpus from web-based texts. One reason why,
in general, is that the inclusion of frequently repeated text items, such as those appearing in site 
navigation menus, can lead to misleading statistics. For parallel text in particular, the inclusion of a
language-choosing element can itself lead to misleading translations: the spot offering ``B\'earla'' in the
Irish version of a website would quite naturally offer ``Irish'' in the same location of the English version
of the page, leading to an incorrect pairing of ``B\'earla'' and ``Irish''.

The high quality page generation of the Citizen's Information site again provides assistance. On each generated
page, the content of the page is separated clearly from the boilerplate with cues in the HTML markup:

\begin{verbatim}
<!-- start of Document -->
<a name="startcontent"></a>
\end{verbatim}

One of the most attractive qualities of the Citizen's Information website is that it is one of the
first government-run websites to fully embrace the recommendations on the Re-use of Public Sector Information,
with all content of the site available under the recommended Open Content licence, Creative Commons CC-BY 4.0.
The terms of this licence allow content that is covered by it to be re-used, modified, and/or redistributed,
as long as an appropriate acknowledgement of the source of the information is provided. This permissive
licence means that any linguistic resources resulting from the website can be used at will, without relying
on research exemptions, and can therefore be used by commercial entities as well as by other researchers.

Finally, that the content of the Citizens Information site is available as HTML is an advantage. Revenue.ie,
the Irish Tax and Customs website, is also available under an open licence, with a large amount of parallel
text in English and Irish, but the majority of its documents are available as PDF. HTML, as a format designed
for on-screen display, is much easier to align at a document level: for example, paragraphs are marked
semantically, with the decision of how they ought to be rendered left to the web browser -- the site designer
may set a number of properties using style sheets, etc., to control how they are presented, but these
properties can be ignored, which is quite often desirable when the reader is using technological solutions
for accessibility, such as screen readers for the blind.

PDF, on the other hand, is designed for print layout. Although PDF can additionally contain the plain text
that is visible within the document, it is not always the case that it is present; even when present, the
PDF creation software may have used its own internal encoding system for rendering characters, such as vowels
with fadas; also, if the document was not ``digital native'', but created through scanning, the text may have
been added using Optical Character Recognition (\GLS{OCR}), which may introduce errors of its own: a common
example is ``arid'' in place of ``and'', due to the visual similarity of \texttt{ri} and \texttt{n}.

\section{Conversion}

Because of the relatively small amount of processing required, due to the nature of the Citizen's 
Information website, much of the usual processing involved in preparing a bilingual corpus from the web
could be bypassed: documents were paired based on their address, language content checked based on that
reported in the \texttt{<html>} tag, and a simple script (included in the supplemental data repository, and
is also reproduced in Appendix~\ref{app:bpremoval}.

\subsection{Phrase based training}

For the phrase based models, because a comprehensive examination of the translation performance on various
parallel corpora was available, and because of the poor quality of a significant portion of that data,
I decided to compare the existing public results with the result of using the cleaning script that
accompanies the Moses decoder, which removes sentences where the ratio of words is 9:1 or greater. 
As well as that, knowing that the oddity of case folding in Irish would
mean that using the standard lower casing method would lead to the presence of a second, artificial form
of a number of words, I decided to additionally compare the performance of systems trained with and
without correct case folding. The script used to do so is relatively uncomplicated, and is reproduced in 
figure~\ref{fig:tolowersh}. Moses~\citep{Koehn:2007:MOS:1557769.1557821} was used for all SMT experiments.

\begin{figure}
\begin{verbatim}
#!/bin/bash
TRANSLIT=$(cat <<'__HERE__'
$wordBoundary = [ [:Z:] [:P:] ];
::NFD();
$wordBoundary { n } [AEIOU] > n\- ;
$wordBoundary { t } [AEIOU] > t\- ;
::Any-Lower;
::NFC();
__HERE__
)

uconv -x "$TRANSLIT"
\end{verbatim}
\label{fig:tolowersh}
\caption{Script to perform correct lowercasing of Irish.}
\end{figure}

\subsection{Factored models}

The same corpus was used for factored models as phrase-based; the data that was part of speech tagged was the cleaned version,
with the Irish lowercased with the Irish-specific method.

\subsubsection{Processing}

Despite the availability of rule-based tools for the morphological analysis and tagging of Irish, IrishFST~\citep{elain09}, 
I chose to train a set of statistical models for the tagging and lemmatisation of Irish. 
The primary motivation was to use a method where a single tagged wordform was selected, correct or not -- the constraint
grammar used to disambiguate the output of IrishFST makes no guarantee that its output will be unambiguous, which poses
a difficulty in systems that expect a single tagged output form.

A secondary motivation was to follow the workflow most amenable to practitioners of statistical NLP, who tend to prefer not 
to use rule-based tools where possible, albeit with certain exceptions -- for example, the sentence splitting and word tokenisation
tools provided originally by the Moses SMT toolkit, which are widely used by other MT and NLP systems, are quite
decidedly rule-based: the sentence uses a list of known abbreviations to avoid incorrect sentence breaks, much like
the components used in Translation Memory systems, which typically use Segmentation Rule eXchange (\GLS{SRX})~\citep{milkowski09srx}.

The options for the statistical processing of Irish are somewhat limited. The only publicly available
model that I was able to find was for Google's SyntaxNet dependency parser, which was trained on the
data from the Universal Dependencies project~\citep{lynn2016universal}. As this dataset is quite small,
I chose instead to train both a part of speech tagging model and lemmatisation model on the Irish Dependency Treebank~\citep{lynn16treebank},
using Apache  OpenNLP\footnote{\href{https://opennlp.apache.org/}{https://opennlp.apache.org/}}, primarily because of prior
familiarity, although OpenNLP as a toolset is designed with a quite broad-minded approach to NLP: most of
the standard facilities are statistically oriented, but several co-exist with rule-based equivalents.

A number of corrections were applied to the tagging of the corpus, primarily to repair the output of
the guesser module of IrishFST, both to reduce the size of the tagset, and to repair its 
lemmatisation\footnote{The corrected version is available online: \href{https://github.com/jimregan/IrishDependencyTreebank}{https://github.com/jimregan/IrishDependencyTreebank}} -- 
most of the words with guessed tags were compounds, for which the lemma of the head of the compound was
given in place of a lemma for the whole.

\subsubsection{Multiwords}

The Irish Dependency Treebank, having been initially tagged with IrishFST, makes use of its tokenisation, and therefore
contains a number of tokens that represent multiword entities. As these cannot easily be reproduced in
other systems, the individual words that these multiwords are composed of needed to be retagged 
individually.

Splitting the multiwords into their individual words has posed some difficulties: in a number of cases, 
the multiwords contain calcified words which no longer have meaning in modern Irish. For example, 
``moite'' in the phrase ``c\'e is moite'' (\textit{except}) is absent from dictionaries of the 
modern language, except as part of that phrase\footnote{eDIL (\href{http://www.dil.ie/32548}{http://www.dil.ie/32548}) lists it as a comparative of ``m\'or''}, 
and ``dt\'i'' in the phrase ``go dt\'i'' is a calcified present subjunctive form of the verb ``tar'', 
but can legitimately be analysed in the modern language as the eclipsed forms of two nouns which 
share the lemma ``t\'i'': a feminine noun meaning \textit{line} or \textit{track}, and a masculine 
noun meaning \textit{tee} (as in golf). These were retagged along linguistic lines, both because
of the lack of an acceptable alternative, and to not introduce confusion for similar contexts: in
these examples, of a comparative adjective after ``is'', and of a subjunctive following ``go''.

The names of political parties are typical of the opposite problem: for example, ``Sinn F\'ein'' is
composed of two pronouns, yet behaves as a single (masculine) proper noun. In such cases, both
components were marked with the tags of the whole.

Even though it is not a meaningful metric to evaluate a statistical NLP tool on the data it was trained on,
where it can be expected to achieve misleadingly high levels of accuracy, the reported accuracy of 99.4\%
seemed, if anything, somewhat low. OpenNLP's evaluation tool can report on which words failed to meet
expectations, and a quick examination was somewhat informative: although in large part, the errors were 
genuine, such as in the phrase ``as ascaill a mháthar'', where for ``a'', OpenNLP predicted the infinitival 
particle, rather than the masculine possessive determiner (meaning \textit{his}), there were some ``errors''
which were not errors on OpenNLP's account, but rather errors in the training data: for example, in the phrase
``maidir le bhfíorú'', ``le'' in the corpus was mistagged with \texttt{Prep:Poss:3P:Pl} -- which cannot be 
generated by IrishFST -- rather than the correct \texttt{Prep:Simp}, which OpenNLP correctly predicted. 
Similarly, in ``uair an chloig'', the word ``chloig'' in the corpus was mistagged with 
\texttt{Noun:Masc:Gen:Sg:Len}, rather than the correct \texttt{Noun:Masc:Gen:Sg:DefArt}, which OpenNLP also 
correctly predicted.

As OpenNLP has the option to include a tag dictionary, an attempt was made to include the entire output
of IrishFST as a tag dictionary; however, this attempt was unsuccessful because of the extremely high
number of tags (3098) that are present in IrishFST but absent from the training data.




\section{Neural Machine Translation}
\label{sect:ch3nmt}

The data for Neural Machine Translation was processed initially in the same way as the data for factored models,
then tokenised using BPE~\cite{sennrich2015neural}. Training and translation was performed using the OpenNMT~\citep{2017opennmt}
toolkit.