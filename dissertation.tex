%%%%%
%%
%% Sample document ``thesis.tex''
%%
%% Version: v0.2
%% Authors: Jean Martina, Rok Strnisa, Matej Urbas
%% Date: 30/07/2008
%%
%% Copyright (c) 2008-2011, Rok Strniša, Jean Martina, Matej Urbas
%% License: Simplified BSD License
%% License file: ./License
%% Original License URL: http://www.freebsd.org/copyright/freebsd-license.html
%%%%%

% Available documentclass options:
%
%   <all `report` document class options, e.g.: `a5paper`>
%   withindex   - enables the index. New index entries can be added through `\index{my entry}`
%   glossary    - enables the glossary.
%   techreport  - typesets the thesis in the technical report format.
%   firstyr     - formats the document as a first-year report.
%   times       - uses the `Times` font.
%   backrefs    - add back references in the Bibliography section
%
% For more info see `README.md`
\documentclass[withindex,glossary,times]{tcd-dissertation}

% Citations using numbers
\usepackage[natbibapa]{apacite}
\bibliographystyle{apacite}
\usepackage{fontspec}
\usepackage{epigraph}
\usepackage{xeCJK}
%\setCJKmainfont{SimSun}

%%%%%%%%%%%%%%%%%%%%%%%%%%%%%%%%%%%%%%%%%%%%%%%%%%%%%%%%%%%%%%%%%%%%%%%%%%%%%%%%
%% Thesis meta-information
%%

%% The title of the thesis:
\title{English-Irish Machine Translation}

%% The full name of the author (e.g.: James Smith):
\author{Jim O'Regan}

\degreetitle{M.Phil in Speech and Language Processing}
\degreeyear{2017}

\supervisor{Dr. Elaine U\'i Dhonnchadha}

%% Submission date [optional]:
% \submissiondate{November, 2042}

%% You can redefine the submission notice [optional]:
% \submissionnotice{A badass thesis submitted on time for the Degree of PhD}

%% Declaration date:
\date{September, 2017}

%% PDF meta-info:
\subjectline{Irish-English MT}
\keywords{MT, Irish, English}

\foreword{%
This dissertation was supported by funding from The Department of Arts, Heritage, 
Regional, Rural and Gaeltacht Affairs (now The Department of Culture, Heritage and the 
Gaeltacht).

Chapter 2 incorporates text from my contribution to the article ``\guillemotleft All 
Your Data Are Belong to us\guillemotright . European Perspectives on Privacy Issues in 
`Free' Online Machine Translation Services''~\citep{kamocki2015all}, which was carried
out during the period of registration and integrated into the body of the dissertation,
according to the guidelines issued by the Office of the Dean of Graduate Studies,
February 2017, and as discussed with my supervisor.

As this work was created using public funding, and in the interest of openness, I
release this work under the terms of the Creative Commons Attribution 4.0 International 
(CC BY 4.0) licence. As the copyright holder, this notice supercedes any conflicting
notice that may subsequently be put in place by the library of Trinity College, Dublin.
Where such a conflict might occur, the reader is authorised to choose the least 
restrictive of the two.
}

%%%%%%%%%%%%%%%%%%%%%%%%%%%%%%%%%%%%%%%%%%%%%%%%%%%%%%%%%%%%%%%%%%%%%%%%%%%%%%%%
%% Abstract:
%%
\abstract{%
This dissertation documents the development of a Rule-Based Machine Translation
system for English to Irish, describing the conversion of a pair of dictionaries
for use as a bilingual lexicon, and the conversion of an existing morphological
analyser for analysis and generation.

It describes a tool created for creating translation rules, based on the alignment
templates used in Statistical Machine Translation. The tool is only partially
complete, as it was intended to also allow the specification of rules with
partial recursion, to reduce the amount of rules that required to be written.
Due to the demands of the English-Irish language pair, rules of a single element
are in the minority; poor foresight on the authors part, along with the complexity
of transforming multiple rule pieces that must remain together, have resulted in
the partial completion of this tool. Consequently, the rules for the English-Irish
pair which were to be described here are not discussed, pending manual expansion
from the form they were written in, which expected this recursive expansion. 

An intended comparison with Statistical Machine Translation and Neural Machine 
Translation was additionally hampered by the poor quality of much of the
available parallel text. This has resulted in the creation of a proof of concept
tool for corpus cleaning, described within.
}




%%%%%%%%%%%%%%%%%%%%%%%%%%%%%%%%%%%%%%%%%%%%%%%%%%%%%%%%%%%%%%%%%%%%%%%%%%%%%%%%
%% Glossary [optional]:
%%
\newglossaryentry{MT}{
    name=MT,
    description={Machine Translation}
}
\newglossaryentry{EBMT}{
    name=EBMT,
    description={Example-Based Machine Translation}
}
\newglossaryentry{NMT}{
    name=NMT,
    description={Neural Machine Translation}
}
\newglossaryentry{RBMT}{
    name=RBMT,
    description={Rule-Based Machine Translation}
}
\newglossaryentry{SMT}{
    name=SMT,
    description={Statistical Machine Translation}
}
\newglossaryentry{HMM}{
    name=HMM,
    description={Hidden Markov Model}
}
\newglossaryentry{ANN}{
    name=ANN,
    description={Artificial Neural Network}
}
\newglossaryentry{POS}{
    name=POS,
    description={Part of Speech}
}
\newglossaryentry{FST}{
    name=FST,
    description={Finite State Transducer}
}
\newglossaryentry{WFST}{
    name=WFST,
    description={Weighted Finite State Transducer}
}
\newglossaryentry{GPU}{
    name=GPU,
    description={Graphics processing unit}
}
\newglossaryentry{TM}{
    name=TM,
    description={Translation Memory}
}
\newglossaryentry{SRX}{
    name=SRX,
    description={Segmentation Rule eXchange}
}
\newglossaryentry{OCR}{
    name=OCR,
    description={Optical Character Recognition}
}
\newglossaryentry{ASR}{
    name=ASR,
    description={Automatic Speech Recognition}
}
\newglossaryentry{RNN}{
    name=RNN,
    description={Recurrent Neural Network}
}
\newglossaryentry{LSTM}{
    name=LSTM,
    description={Long Short-Term Memory}
}




%%%%%%%%%%%%%%%%%%%%%%%%%%%%%%%%%%%%%%%%%%%%%%%%%%%%%%%%%%%%%%%%%%%%%%%%%%%%%%%%
%% Contents:
%%
\begin{document}

%%%%%%%%%%%%%%%%%%%%%%%%%%%%%%%%%%%%%%%%%%%%%%%%%%%%%%%%%%%%%%%%%%%%%%%%%%%%%%%%
%% Acknowledgements:
%%
\acknowledgements{%
To my supervisor, Elaine U\'i Dhonnchadha: I know the risk to your reputation
that you took when you accepted me into this programme, and internship, and
that the results documented here cannot vindicate that decision is a bitter,
crushing disappointment. I hope to rectify this as soon as possible. 

To the other members of the 
academic panel: Breffni O'Rourke, Carl Vogel, Christer Gobl, and Ailbhe 
N\'i Chasaide, I am grateful for your roles in making this possible,
and to Kevin Scannell, Piotr Ba\'nski, and Mikel L. Forcada, I am grateful
for the letters of recommendation, and the untiring work each of you do
for the advancement of open source in language technology and resources.

I gratefully acknowledge the support I've received from the Department of 
Culture, Heritage and the Gaeltacht, and thank the Department for it;
without which this work would not have been possible. In particular, 
I'd like to thank Aodh\'an Mac Cormaic: I could not imagine a better
representative for the Department.

To my parents, Joe and Julie: thank you for making me what I am, and for your
support, even at the most trying of times. I couldn't have asked for better
parents.

To my son Mark: you have been my reason to keep going, even when times were
hardest.

To my siblings: Joe, Paul, Angela, Anne Marie, and Catherine: thanks for 
being you. To Joe in particular, thank you for the use of your computer,
without which I could not have completed much of the work I've written about
here. To Anne Marie, as well as my friends James Doyle and Peiyao Li, thank
you for listening to my whining when things were yoyoing out of control.

A debt of gratitude is owed to Joanna Kozicka, for the continuous reminders
that there is a life on the other side of this process. A serious thank
you for sharing your stories -- listening to them is the only thing that
kept me from going completely insane.

The Phonetics and Speech Lab in Trinity has been like a second home for the
past two years, and its other occupants a second family. To M\'iche\'al \'O 
Meachair, Emily Barnes, Fionnuala Nic Ph\'aid\'in, Christoph Wendler, and
Neasa N\'i Chiar\'ain, thank you for your immense help in understanding the
often vague, and frequently underdocumented, aspects of Irish grammar.
Eoghan O'Connor, Andy Murphy, and Maria O'Reilly, thanks for the crosswords.
Jenny O'Reilly, thank you for saving me -- you know how, but it's probably
best not to put it in writing! To Antoin Rodgers (and Eoghan, and James), 
thank you for saving me from so many nights where I could easily have ended
up sleeping rough or roaming the streets. And to all, thanks for the 
doughnuts, the beer, and the many PhLabby memories. (But also, a special
``no thank you'' to Emily for the nickname ``Jimmington'', and my sisters
for adopting it).

To Beata Kozyra: your inspiring me to learn Polish indirectly lead to my
first steps with Natural Language Processing, and watching you painfully
and painstakingly translate a manual, word by word, is what directly lead to my
interest in Machine Translation. Though ours paths may never again cross,
thank you for leading me to the path I'm now following. To Mariusz Florczak
and Tomasz Mikulicz, whom I met on that path, thank you for your help and
your friendship.

Thanks also to my classmates, department colleagues and fellow students over the past two
years. In particular, to Mari Palacios, Vangelis Dimos, Camilla Elan, 
Daniela Modrescu, and Anshu Dash: thanks for your encouragement and support.

A special thank you to Mihael Ar\v{c}an, developer of the IRIS system, and
to Meghan Dowling of Tapad\'oir, for being so keen to share information on
English-Irish Machine Translation.

To the many members of the various open source projects whose software I have
used, and to those with whom I've had the pleasure of collaborating, I thank 
you for your efforts. In particular, the community around Apertium helped
me to get to this point: a special thank you to Fran Tyers, Gema Ram\'irez-S\'anchez,
Sergio Ortiz, Jacob Nordfalk, Kevin Unhammer, Mireia Ginest\'i Rosell, and
Felipe S\'anchez Mart\'inez, as well as to the many students I had the pleasure
of working with as part of Google's Summer of Code and Code-In programmes. I
would here also like to thank my friend and co-author, Pawe\l{} Kamocki, whose
contribution of legal advice does not get quite the amount of recognition it
deserves.

Finally, to ``Oliwka N.'', who (unwittingly) supplied the epigraphs to my chapters:
meeting you was the highest point of the last two years. Muchissimo!

}

%%%%%%%%%%%%%%%%%%%%%%%%%%%%%%%%%%%%%%%%%%%%%%%%%%%%%%%%%%%%%%%%%%%%%%%%%%%%%%%%
%% Title page, abstract, declaration etc.:
%% -    the title page (is automatically omitted in the technical report mode).
\frontmatter{}


%%%%%%%%%%%%%%%%%%%%%%%%%%%%%%%%%%%%%%%%%%%%%%%%%%%%%%%%%%%%%%%%%%%%%%%%%%%%%%%%
%% Thesis body:
%%
%\def\longs{{\fontencoding{TS1}\selectfont s}}
\chapter{Introduction}

% The TODO for this chapter
This introduction will provide a high level overview of Machine Translation (\GLS{MT}) (section~\ref{sect:intromt}), and of the 
particular challenges posed by Irish (section~\ref{sect:introirish}).
It will also describe briefly the approach to evaluation 
which will be employed (section~\ref{sect:introeval}), and introduce the structure of the remainder of the dissertation.



\section{Machine Translation}
\label{sect:intromt}

Machine Translation (MT), or automatic translation, is an automated process that recodes one human language 
into another. Several approaches exist to tackling the problem, which can be classified in a number of 
ways: of those considered in this dissertation, the primary distinction is between knowledge-based (linguistic) approaches, 
using dictionaries and grammatical rules to perform the translation; and data-driven approaches, which 
learn to translate using a corpus of existing translations.

Machine Translation is used for two primary purposes: \textit{assimilation}, to get the gist of text in 
a foreign language, and \textit{dissemination}, as an input to publication, typically post-edited by 
translators. Free online services, such as Google Translate, are usually intended for assimilation, 
while the translation services in use at multilingual organisations such as the European Union are 
typically intended for dissemination. Consequently, systems can be designed with certain 
trade-offs in mind: for example, a system intended primarily for assimilation may trade accuracy for 
broader coverage, and vice versa.

The amount of data and effort required to build an MT system for either task depends on a number of 
factors, first among which is how closely-related the languages are. More closely-related languages 
tend to share similar structures, and tend to have more cognate words (which, even when unrecognised 
by the translator, pass through as ``free rides''). \cite{trosterud2012evaluating} compare a gisting 
system from North S\'ami to Norwegian Bokm{\aa}l (non-Indo-European to Indo-European) with their earlier 
work on Norwegian Nynorsk and Bokm{\aa}l~\citep{unhammer2009rfr}, which achieves near post-edition quality 
using less than a third of the amount of transfer code. 

Machine Translation, ultimately, is a tool for facilitating communication, and the needs of communication 
tend to dictate the purpose: when translating from a major language to a minor language, assimilation 
is the primary need of the majority of (potential) users, whereas when translating from a major language 
to a minor language, the primary need is dissemination, to gain more text in the minor language.

\subsection{Technology and challenges}

Knowledge-based, or Rule-Based Machine Translation (\GLS{RBMT}), systems can be classified into three main categories: \textit{direct translation}, where no 
transformation of the source text is performed; \textit{transfer-based}, where the input is transformed 
into a language-dependent intermediate representation -- typically, \textit{lemma}: the citation 
form of the word; \textit{morphological analysis}: details such as case, number, person, etc.; and 
possibly \textit{semantic analysis}: the subject of a verb, etc. -- and \textit{interlingua}, where the 
input is transformed into a language independent representation. 

Rule-based systems tend to be costly and time-consuming to build, as they require lexicographers and 
linguists to create dictionaries and rules. Transfer-based and interlingua approaches, as they operate 
on abstract representations, generalise better -- if a word is in the dictionary, it can be handled in 
all forms -- but they tend to handle exceptions, such as idioms, poorly: specific rules must be written, 
and multiple rules may conflict with each other. Additionally, the complexity of an interlingua tends,
in reality, to increase with the addition of each additional language, as each new language adds its
unique terms and structures, the existing lexical and syntactic rule base needs to expand to 
accommodate them.

The prerequisite for building data-driven systems, including Statistical Machine Translation (\GLS{SMT}),
Neural Machine Translation (\GLS{NMT}), and Example-Based Machine Translation (\GLS{EBMT}), is the 
existence of human-translated bilingual (or multilingual) corpora.

A common source of professionally translated multilingual corpora are international organisations such 
as the United Nations or the European Union, which generate a substantial amount of freely available, 
high-quality multilingual documents (in 24 languages for the EU\footnote{Art. 1 of the Regulation No. 1 
of 15 April 1958 determining the languages to be used by the European Economic Community.} and in 6 
languages for the UN\footnote{Rule 51 of the Rules of Procedure of the General Assembly of the United 
Nations; rule 41 of the Provisional Rules of Procedure of the United Nations Security Council.}).

Compared to rule-based MT systems, data-driven MT systems are cheaper -- at least, for widely-spoken 
languages -- and more flexible: a data-driven system is not designed specifically for one language 
pair, but can accommodate any language pair for which a corpus is available. Also, because data-driven 
MT systems are based on human-translated texts, the output of such systems is, or, at least, can be, 
more natural (or ``fluent''), and it naturally adapts well to exceptions: if the corpus contains the 
phrase, it is effectively not an exception.

\textit{Zipf's law} states that in a given corpus, the frequency of a word is inversely proportional 
to its frequency rank: the most frequent word will occur (approximately) twice as often as the second, 
three times as often as the third, and so on. Conversely, the majority of words (40-60\%) are 
\textit{hapax legomena} (words which only occur once). As data-driven MT is corpus-based, it therefore 
suffers from the problem of \textit{data sparsity} due to the high proportion of hapax legomena: longer 
phrase matches are absent from the translation model; contextual information is absent from the language 
model, affecting the quality (``fluency'') of the output.

Data sparsity is the biggest problem in data-driven MT. Although there have been attempts to solve it 
by using linguistic information, dating back to CANDIDE (the first SMT system), the most common approach 
is to simply add more data\footnote{On the other hand, it has been claimed that translation quality can be 
increased by simply discarding infrequent phrases: \citet{Johnson07improvingtranslation}.}. A large 
amount of websites are available in multiple languages, so crawling the web for parallel text is a 
common method of collecting corpora~\citep{zora80038}, particularly for the providers of free online 
MT, such as Google and Microsoft, who also operate search engines and therefore already have access to 
such data. The use of such data, however, has its own problems, as such documents are often not just 
translated, but \textit{localised}: different units of measurement, currency, and even country 
names~\citep{quince08}, because of their collocation, become ``translations''. 

Crowdsourcing, where online communities are solicited for content, is also in increasing use. Amazon's 
Mechanical Turk, an online market place for work, provides an easy way to pay people small amounts of 
money for small units of work, which has been used in MT~\citep{zaidan2011}. Google has made use of 
crowdsourcing since the earliest days of Google Translate, by providing users with a means of improving 
translation suggestions~\citep{chin07suggest}, by providing a translation memory 
system~\citep{galvez09toolkit}, and more directly, with Translate Community~\citep{kelman14community}. 
Crowdsourcing has been recently used for Irish-English MT, via user contributed translations of 
tweets~\citep{dowling2017crowd}.

Finally, the quality of MT output depends on the quality of the input. Even the most banal imperfections, 
such as misspellings or grammar mistakes, even if they are barely noticeable to a human translator, can 
compromise the most elaborate MT systems~\citep{porsiel12sec}.

\section{Irish}
\label{sect:introirish}

Irish is an Indo-European language, from the Q-Celtic branch of Celtic languages, closely related to 
Scots Gaelic and Manx, and more distantly related to Welsh, Breton, and Cornish. Irish exists in a 
somewhat paradoxical state in that, while it is ``definitely endangered''~\citep{moseley2010atlas}, it
is also the first national language of the Republic of Ireland, and is an official language of the
European Union; although Irish is a compulsory subject for all students\footnote{A limited set of
exemptions are outlined in Circular letter M10/94: 
\href{https://www.education.ie/en/Circulars-and-Forms/Active-Circulars/ppc10_94.pdf}{https://www.education.ie/en/Circulars-and-Forms/Active-Circulars/ppc10\_94.pdf}.},
30.1\% of 10-19 year olds interviewed for the 2016 census answered ``no'' to the question ``can you
speak Irish?''\footnote{\href{http://www.cso.ie/en/media/csoie/releasespublications/documents/population/2017/7._The_Irish_language.pdf}{http://www.cso.ie/en/media/csoie/releasespublications/documents/population/2017/ 7.\_The\_Irish\_language.pdf}};
and despite the existence of designated Irish speaking, or \textit{Gaeltacht}, areas, the same census
reports that only 32\% of the Irish speakers living in Gaeltacht areas speak Irish on a daily basis.

There are three primary dialects of Irish, corresponding to three of the four provinces of Ireland:
Connacht, Munster, and Ulster (there is evidence of a Leinster dialect having existed); there is
a quite low level of mutual intelligibility between the dialects, though this is reducing thanks
to the availability of television and radio in Irish. For the most part, the differences are most
evident in the spoken language, although differences in syntax do exist. A written standard for
Irish has existed since the 1950s, though as it is intended solely to provide a written standard
for official publications, and ``not an official standard for speech nor a denial of the linguistic wealth of the various dialects''
(``Ní caighdeán oifigiúil le haghaidh na cainte atá i gceist ná ní séanadh ar an saibhreas teanga sna canúintí éagsúla é.''~\citep[xviii]{caighdean2017}).

Unlike most other European languages outside of the Celtic family, Irish is a VSO language,
with adjectives and other modifiers typically following the noun or noun phrase.
Irish is an inflected language, with suffix or internal inflection, as well as a set of initial
mutations which are typical of Celtic languages. Nouns have gender -- masculine or feminine --
and inflect for case and number. Nouns can have three cases: a common nominative/accusative,
genitive, and vocative. There are also calcified remnants of a dative case, which is still
in use in some areas.

Verbs inflect for tense and mood, and may, particularly in the Munster dialect, also inflect for person
and number. Verbs are quite
regular, although a small number of the more common verbs are extremely irregular. Irish has
two equivalents of the verb ``to be'': the existential verb ``b\'i'' and the copula ``is'', 
which share the same Indo-European origins as the English ``be'' and ``is'', or Polish 
``by\'c'' and ``jest''. Irish has no single equivalents of ``yes'' or ``no'', using instead
the repeated verb (or copula) or its negation to express an affirmative or negative response.

\subsection{Problems facing Irish}

\subsubsection{Lack of linguistic research}
\label{sssect:lingresearch}

``In the area of syntax, basic linguistic research on the linguistic phenomena of Irish syntax is 
required''~\citep[p. 73]{LWP-Irish}. One such example is reflexive verbs and verb phrases in Irish.
While some study exists~\citep[e.g.,][]{nolan2004rrg,nolan2000reflexive}, they focus mainly on the use of the
reflexive pronoun, ``f\'ein'', while there are other constructs which show at least basic reflexivity, such
as verbs which express their object with an adjunct: ``chuir m\'e mo ch\'ota orm'' (\textit{I put my coat on 
(myself)}, ``Brostaigh ort!'' (\textit{Hurry (yourself) up}).

Similarly, the infinitive of a verb is
expressed with a verbal noun preceded by the infinitival particle ``a''; however, in constructs where
the object of a verb or copula phrase is an infinitival phrase, the infinitival particle is dropped
where there is no object of the infinitival verb: e.g., ``is maith liom leabhar a l\'eamh'' 
(\textit{I like to read a book}) retains the infinitival particle, but 
``is maith liom l\'eamh'' (\textit{I like to read}) does not, but grammars of Irish do not typically
speak of this as being the same process; for example, \citet[p. 128]{grammar1980cb} explicitly mention
in point 7 that the negative infinitive is expressed with the preposition ``gan'' and the verbal noun,
while indirectly mentions in point 9 that the equivalent when there is an object uses the particle ``a'',
but without linking the two.

\subsubsection{Technological support}

\citet[s. 5.1]{krauwer2003basic} defines ``the  Basic  Language  Resource  Kit  
(abbreviated  BLARK)  as  the  minimal  set  of  language  
resources that is necessary to do any precompetitive 
research and education at all'', and includes in it linguistic resources such as dictionaries, 
grammars, and corpora, as well as tools such as text to speech, morphological analysers, taggers,
parsers, and speech recognisers.

Although Irish is considered to have only ``fragmentary support'' in the area of speech processing,
and ``weak/no support'' in the areas of machine translation, text analysis, and speech and text
resources~\citep[pp. 77--78]{LWP-Irish}, there is a perhaps surprising range of technological support,
given the relatively low number of Irish speakers.

For basic text processing purposes, support is available for both spell 
checking, through GaelSpell\footnote{\href{https://borel.slu.edu/ispell/}{https://borel.slu.edu/ispell/}}, and 
grammar checking, through \textit{An Gramad\'oir}\footnote{\href{http://borel.slu.edu/gramadoir/}{http://borel.slu.edu/gramadoir/}}\footnote{Another
grammar checker, based on the ``LanguageTool'' language checker described by \citet{naber2003rule}, was
commenced during the early stages of this work, coupling the text analysis resources of
\citet{elain09} with the grammar checking with the rules of \textit{An Gramad\'oir}, but needs
some additional work to be fully integrated.}.

For text analysis, there is a morphological analyser and disambiguator~\citep{elain09},
a statistical part-of-speech tagger for tweets~\citep{lynn2015minority}, 
a rule-based partial dependency parser~\citep{UDHONNCHADHA10.824}, and a
treebank for statistical dependency parsing~\citep{lynn16treebank}, a subset of which
is available mapped to the Universal Dependencies scheme~\citep{lynn2016universal}.

Speech synthesis (text to speech) for all three dialects is available 
online\footnote{\href{http://www.abair.ie}{http://www.abair.ie}}~\citep{chasaide2017abair}, 
which is employed in a prototype chatbot for Computer Assisted Language Learning
(CALL)~\citep{nichiarain2016chatbot}.

In terms of resources, Irish fulfils many of the requirements of a BLARK. Although there
are relatively few dictionaries, those that are available are available in digital form:
``An Focl\'oir Beag''~\citep{odonaill1991focloirbeag}, the only monolingual dictionary
of modern Irish; ``Focl{\'o}ir Gaeilge-B{\'e}arla''~\citep{donaill1977focloir}, the Irish--English
dictionary of reference; and ``English--Irish Dictionary''~\citep{debhaldraithe1959english}, the 
previous English--Irish dictionary of reference have all been digitised and made available online\footnote{http://www.teanglann.ie/ga/},
while the current English--Irish Dictionary of reference is digital native\footnote{http://www.focloir.ie/}; 
in addition to dictionaries, there is a publicly accessible terminology database\footnote{\href{http://www.tearma.ie/}{http://www.tearma.ie/}}~\citep{mvechura2010focal}, and
as well as human-targeted dictionaries, there are lexicons intended for use in computing, including a WordNet for Irish~\citep{oregan2016lemongawn}
and a high-quality morphological database~\citep{mchura2014}.
In terms of corpora, the New Corpus for Ireland~\citep{Kilgarriff2006EfficientCD} contains 30 million words.

\subsubsection{Character encoding}

Although not a problem peculiar to Irish, the issue of character encoding, or, rather, of incorrectly recoded
text, arises unusually often with Irish texts.

In earlier days of computing, a number of 8-bit encoding schemes were employed in various areas around the world,
to allow the local alphabets to be encoded in text. ASCII, the basic set of characters that are sufficient to
encode the English alphabet along with a number of common characters, such as punctuation and mathematical operators,
consists of less than 128 characters, which can fit into 7 bits of computer memory -- a ``bit'' encodes a single
boolean 1 or 0, so a sequence of 7 holds $2^7$, or 128\footnote{To be clear, 0 to 127}, possibilities. As the computers of the time were typically
able to address 8 bits (one byte) of memory at a time, this left one extra bit, allowing a further 128 characters to be used,
which were typically arranged among the dominant languages of neighbouring countries: Latin1 served the Latin-script
based languages of Western Europe, including the accented vowels used by Irish.

As computers became more powerful, and able to address larger amounts of memory, a coding scheme named Unicode was
with the goal of serving all of the world's languages. Because each character took more than 8 bits to encode,
a number of standardised methods for serialising these characters were devised: the most common of which, UTF-8,
is a \textit{variable-width encoding}, meaning that the higher the number representing the character was, the
more individual bytes were required to represent it. One particular advantage of UTF-8 is that the set of ASCII
characters remain unchanged, with the eighth bit serving to represent that the preceding byte was part of the
same Unicode character. All of the accented vowels used in modern Irish require two bytes, while the remainder
require one. When the characters are incorrectly recoded -- that is, when characters in, for example, UTF-8 are
misinterpreted as having been encoded in Latin1 -- these two bytes are displayed as two individual incorrect
characters: for example, ``{\'e}'' will be recoded as \texttt{é}.


\subsection{Particular challenges of Irish}

While most languages have aspects that can be challenging for particular types of technology, there
are a number of aspects of Irish that are particularly challenging for all language-related processing.

\subsubsection{Numbers}

Numbers in Irish are more complicated than in English; cardinal numbers have a slightly different form
for use when counting, and numbers distinguish between human and non-humans in agreement for numbers
between two and ten, along with twelve.

Numbers in Irish can trigger a mutation in the noun that follows; for example, ``dh\'a'' (\textit{two})
triggers lenition, while ``seacht'' (\textit{seven}) triggers eclipsis. This has posed a problem in
the past for software localisation in systems that were designed with Western European languages in
mind, that could only distinguish between singular and plural, if at all. Irish is far from the most
demanding in its numeric needs -- Slovenian distinguishes between singular, dual, and plural, as well
as the common Slavic genitive of quantity (five or more) -- but has benefited from the needs of other
languages.


\subsubsection{Case folding}
\label{ssect:casefold}

Case folding, in scripts such as the Latin script used by both Irish and English that distinguish 
between upper, lower, and/or title case forms of letters, ``is the process of making two texts identical 
which differ in case but are otherwise `the same'''\citep[s. 2.1]{Phillips:16:CMW}. The conversion from 
upper or title case to lower case is typically referred to as ``lowercasing'' or ``decapitalisation''); 
conversely, the conversion from lower case to uppercase as ``uppercasing'' or ``capitalisation''.

Case folding is important in software that needs to compare text, as the number of combinations in the 
simplest case of a single upper case and single lower case letter is $n^2$, where $n$ is the length of 
the word. For example, the  word ``dog'' has three letters, and yields $3^2 = 8$ combinations: dog, dOg, 
dOG, DOG, Dog, DOg, doG, and DoG.

In English, it's a quite straightforward process to map from upper to lower case by mapping the letters 
from A to Z to the letters from a to z. In other languages, the process is less straightforward: for 
example, the Greek letter sigma ($\Sigma$) has two lower case forms: $\varsigma$ is used only when in 
word final position, while $\sigma$ is used otherwise\footnote{The same was formerly true of English, 
with `s' appearing only in word final position, with the ``medial s'', ſ, used otherwise}. In Dutch, the 
digraph `ij' is considered a single letter ``and thus if a word starts with ij and has to be capitalized, 
both the \textbf{i} and \textbf{j} are affected'', such as in ``IJsland'' 
(\textit{Iceland})~\citep[p. 14]{donaldson2008dutch}.

In Irish, initial mutations (aside from lenition) precede the word proper, and are considered separate
from it: the eclipsed form of the word ``geata'' (\textit{gate}) is ``ngeata'', which is capitalised not
as ``NGEATA'', but as ``nGEATA''. The affixation of a case-invariant part is not unusual: the Zulu 
language uses such prefixes to ``denote proper names endowed with morphological structure as is customary 
in Zulu''~\citep[p. 143]{bosch2011towards}, such as ``isiZulu'' (\textit{the Zulu language}), where the 
case invariant prefix ``isi-'' typically denotes that what follows is a language. It's not uncommon to
see even in English, where a case invariant ``e'' can be used to denote that something is ``electronic'',
or, in possibly the most common instance in current times, in the brand names of Apple's consumer
electronic devices: the iPhone, iPad, and iPod.

The challenge with Irish's prefixes extends with the pre-vocalic mutations, \textit{n}, \textit{t}, and
\textit{h}: \textit{n} and \textit{t} are written in lower case form followed by a hyphen, but without 
when preceding an upper case vowel: ``\'ar n-athair'' (\textit{our father}) is in title case 
``\'Ar nAthair''. Far more complicated is \textit{h}; \textit{n} and \textit{t} have always been part of
the conventional Irish alphabet, and so have always been written in a way that their use as a mutation is
distinct; \textit{h} had traditionally only been used as a prefix, or as a means of marking lenition of the
previous consonant, but modern Irish has imported many words, particularly from Greek, that legitimately
begin with \textit{h}, so to algorithmically convert words beginning with ``h'' to upper or title case 
is not possible without the use of a dictionary\footnote{The closely related Scots Gaelic suffers less
from these problems, as the ``h'' prefix is always written with a hyphen, and the hyphen is retained in
upper case for pre-vocalic mutations.}.

Fortunately, in almost all computing applications, lower case is the norm, which leaves only the 
relatively minor issue of inserting a hyphen between \textit{n} and \textit{t} when their use is as a
mutation.



\section{Evaluation}
\label{sect:introeval}

There are many automatic evaluation metrics for estimating translation quality. All are based on
matching n-grams against a reference translation, though the details vary greatly among the metrics.
As \citet{ARCAN16.9} provide a comprehensive set of evaluations on the range of parallel corpora
available for the English--Irish translation pair, to be able to use their results as a benchmark, I
have opted to use the same evaluation metrics.

BLEU~\citep{Papineni02bleu}, one of the first automatic metrics to be considered comparable with
human judgement, is the de facto standard metric employed when evaluating statistical machine translation
systems. BLEU scores candidate translations against a reference translation, or set of reference
translations, calculating the overlap between the candidate and the reference. The scores for each
segment, or sentence, are averaged across the whole test set, to give a single score.
BLEU's score, although given as a number between 0 and 1 (low to high), is typically reported as a number
between 0 and 100.

METEOR~\citep{denkowski2014meteor} is an automatic metric for translation evaluation that is based on the
harmonic mean of the precision and recall of unigrams, with recall given a higher weight than precision.
Although it has language-specific features for approximate machine, such as stemming and synonyms, these
are supported for a small number of languages (Irish is not included).

chrF~\citep{popovic2015chrf} is a metric for translation evaluation that is based on characters, instead of
words (or, rather, tokens), which is resistant to tokenisation differences, and to slight morphological
errors.

%\section{}
% chap2.tex (Definitions)

\chapter{Background}
\label{chap:litreview}
\epigraph{\textit{\small{It's some kind of this one}}}{``Oliwka N.''}

\section{Introduction}

Knowledge-driven, or rule-based, approaches to Machine Translation are sometimes
contrasted with data-driven, or statistical, approaches by characterising the difference
as that of rationalism versus empiricism~\citep[e.g., ][p. 4]{Somers2003}, although
the difference is not quite so clear-cut: modern knowledge-driven systems are created
with rules and lexicons that are corpus-derived, while data-driven systems typically
include one or more pre- or post-processing stages that are, if not essentially rule-based,
then at least heuristic (often in spite of the fact that statistical equivalents are
available).

Another way of framing the difference can be made in terms of how both approaches scale,
in terms of the exceptional aspects of language: knowledge-driven systems -- particularly
transfer-type systems that use rules based on parts of speech -- scale poorly, as it can be
difficult to encode the irregularites; they do, however, degrade well: novel coinings that follow the
coded syntactic norm will tend to translate well. Data-driven systems, on the other hand,
scale well to the exceptional, as they do not have rules to which there can be an
exception -- there are merely probabilities of words in one language, or sequences of them, being translated
as words in the other. They also, however, tend to degrade quite poorly, as they lack
the knowledge of what could or ought to be the default. There are, as will be mentioned, approaches
that incorporate linguistic information in an attempt to provide this sort of fallback, but
as a general rule of thumb, a statistical system is more likely to know how to ``kick the bucket'',
while a rule-based system will typically be better at handling ``dancing purple elephants''.

The aim of this chapter is to provide the background information necessary to understand
Apertium, the rule-based system used to develop a knowledge-driven machine translation system
from English to Irish; as well as to introduce the ideas which underpin data-driven 
approaches with which it is compared: primarily, Statistical Machine Translation, and Neural
Machine Translation.

Translation Memory, a form of Computer Assisted Translation, deserves mention, if only to
differentiate it from the automatic approaches involved in Machine Translation. At their
simplest, Translation Memory systems segment a document into sentences, and replace those
that have previously been translated with those translations; most will offer suggestions
of incomplete translations, using the same mechanisms as spell checking software, based on
previously translated sentences which have significant overlap with the sentence being edited.

Example-Based Machine Translation (\GLS{EBMT}) is often conflated with Translation Memory. Both
use the sentence as the basic unit of translation, and both appeared at roughly the same time;
however, EBMT is intended as a fully automatic method of Machine Translation, rather than as
a translator's aid: where a translation memory will present an incomplete sentence for the
translator to complete, EBMT attempts to complete the sentence itself. This often involves
the use of linguistic information: in one approach, the parsed version of the corpus is stored
alongside the text, and when a sentence is encountered that differs by a single constituent, if
an equivalent constituent can also be found within the corpus, the EBMT system can reliably
insert that constituent to produce the requested translation. With the advent of approaches
to SMT that involve the use of linguistic data, however, the differences between EBMT and SMT
became fewer, and EBMT approaches were effectively subsumed by SMT.

Hybrid systems, which combine both approaches, will be briefly outlined, in particular those
that combine Apertium with a statistical component, although little mention is made in the
remainder of this dissertation; this is, unfortunately, due to the shift in the basic research
question that had been underlying this dissertation, which had originally been to investigate
the performance of such a hybrid system, but driven primarily by the rule-based component,
rather than using statistical methods to rank output, as is more common. The material is retained
in the hope of anticipating the reader's question ``isn't there some way of getting the best of
both types of system?''

Neural Machine Translation systems are a refinement of Statistical Machine Translation using 
Artificial Neural Networks (\GLS{ANN}), which are the current state of the art in Machine 
Translation. Artificial Neural Networks, as the name suggests, are an attempt to model
the brain's neural networks; although neuroscience remains an important influence, however,
not enough is known about how the brain operates to truly mimic its operation, and modern
neural networks (the ``artificial'' is typically dropped) are inspired by the brain, but
make no claim of copying its actual operation.

\subsection{Finite State Transducers}

Finite state transducers are, or can be, used to model the translation process in both
knowledge-driven and data-driven forms of translation. Finite state transducers are a 
type of finite state machine, that uses two memory tapes: input, and output. In contrast
with a finite state automaton, which uses a single input tape, and can accept or reject
an input, a finite state transducer provides a mapping between input and output.

Finite state machines are in broad use; one example is the controller chip in a computer
keyboard. A keyboard has a number of control keys, in addition to the usual characters of
the alphabet, such as ``Shift'', ''Ctrl'' (Control), etc. When one of these keys is pressed,
the controller enters a different state, causing a different signal to be emitted: when no
control key is pressed, the ``A'' key sends the signal that is interpreted as the lower case
character ``a'', while when the Shift key is pressed, it sends the upper case character ``A''.

A finite state acceptor extends the finite state machine so that there are transitions between
states, and adds a start state, and a set of final states. For example, a lock that accepts
a PIN code will be in the initial state when locked, and no code is being entered. It enters
an accepting state when the first number is entered, and remains in the accepting state while
the code remains valid. Upon completion of the code, if the code is valid, it will enter a
final state, and will unlock. At any accepting state, the input of an incorrect number
will cause it to reject. A spell checker dictionary can be modeled as a finite state acceptor.
By adding input labels to each transition, the checker can output a correctly spelled word by
concatenating the input labels it has accepted, or it can show in an unrecognised word the
portion of the input that caused rejection. 

Such a finite state acceptor can be modeled as
a labeled directed graph, where each state represents a vertex of the graph, and each labeled
transition is a labeled edge between vertices. A finite state acceptor can be formally defined
as the quintuple $(\Sigma, S, s_0, \delta, F)$, where $\Sigma$ is the input alphabet: a finite,
non-empty set of symbols; $S$ is a finite, non-empty set of states; $s_0$ is the initial state,
an element of $S$; $\delta$ is the state-transition function, $\delta: S \times \Sigma \rightarrow S$;
and $F$ is the set of final states, a subset of $S$.

Finite state transducers add labels to each transition for both input and output, which can be
asynchronous, through the use of the empty string, $\epsilon$. A typical use of finite state
transducers is in morphological analysis. For example, the word ``babies'' can be modeled as
\texttt{b:b a:a b:b i:y e:$\epsilon$ $\epsilon$:$<$n$>$ s:$<$pl$>$}, where ``i'' on the input
becomes ``y'' on the output, the epenthetic ``e'' is deleted, the symbol $<$n$>$ inserted, and
the letter ``s'' mapped to the output symbol $<$pl$>$.

Like an acceptor, a finite state transducer can be modeled as a labeled directed graph. One
form of transducer can be modeled as the sextuple $(Q, \Sigma, \Gamma, I, F, \delta)$, where
$Q$ is the set of states; $\Sigma$ is the input alphabet: a finite, non-empty set of symbols;
$\Gamma$ is the output alphabet: also a finite, non-empty set of symbols; $I \subset Q$ is the
set of initial states; $F \subset Q$ is the set of final states; and 
$\delta \subseteq Q \times (\Sigma\cup\{\epsilon\}) \times (\Gamma\cup\{\epsilon\}) \times Q$
is the transition relation, where $\epsilon$ is the empty string. This form of transducer is
known as a \textit{letter transducer}~\cite[pp. 14--63]{roche1997fst}.

Finite state transducers (and acceptors) can be \textit{weighted}, by adding a weight, such
as a corpus derived probability, as an additional label. Many SMT systems extend this further
to a hypergraph structure, where a number of additional labels are added, such as a number of
weights, added by ``feature functions''~\citep{och2002maxent}, and/or linguistic information.

Recurrent neural networks (\GLS{RNN}) have long been used to model finite state acceptors with output that 
are equivalent to finite state transducers~\citep[e.g.,][]{Mikel96beyondmealy}. Most current
work on Neural Machine Translation use an encoder--decoder model~\citep{cho2014rnnencdec},
where the input is transformed to a fixed-length vector, and the output is transformed from
this vector. Most modern work on NMT uses variants of RNN, such as Long Short-Term Memory
(\GLS{LSTM}) that include a memory facility, giving them an expressive power greater than
those of finite-state transducers.


\section{History}
\label{sect:bghist}

Although ideas of how to mechanise the process of translation can be found as early as the seventeenth 
century~\citep[chap. 1]{hutchins1986machine}, Machine Translation as a field of study is typically
considered to have begun shortly after the invention of the digital 
computer~\citep[chap. 1]{koehn2010statistical}. Warren Weaver, a researcher at the Rockefeller Foundation, 
published a memorandum named ``Translation'' in which he outlined the possibility of using computers for 
translation~\citep{hutchins2004weaver}, proposing the use of Claude Shannon's work on Information Theory 
to treat translation as a code-breaking problem -- an approach that would later provide the foundations 
of Statistical Machine Translation (\GLS{SMT}). This memorandum was quite prescient in 
more ways than one: Weaver also cited work by McCullough and Pitts on artificial neural networks, which, 
due to recent advances in ``deep'' neural networks, serve as the basis of Neural Machine Translation, 
the currently dominant research paradigm.

Early research into Machine Translation was concerned with the needs of the Cold War, and researchers 
primarily concentrated on the Russian-to-English and English-to-Russian translation directions, in the 
U.S. and U.S.S.R. respectively. In January 1954 the first public demonstration of an MT system took place 
at IBM's headquarters in Georgetown. In this Georgetown experiment, over sixty sentences were translated 
from Russian to English, leading to a surge of interest in MT~\citep{hutchins2004georgetown}. 

In the years that followed, MT systems were developed by American universities at the behest of organisations such as the U.S. Air Force, Euratom or the U.S. Atomic Energy 
Commission~\citep[chap. 4]{hutchins1986machine}.

In 1964 the U.S. government, concerned about the lack of progress in the field of MT despite significant 
expenditure, commissioned a report from the Automatic Language Processing Advisory Committee (ALPAC). 
The ALPAC report, published in 1966, concluded that MT was unlikely to achieve human-quality translation 
in the foreseeable future~\citep[chap. 8.9]{hutchins1986machine}. 

As a result, MT research was almost abandoned for over a decade in the U.S.; in spite of these 
difficulties, SYSTRAN, a company specialising in MT software, was established successfully in 1968: 
their MT system was adopted by the U.S. Air Force in 1970 and by the Commission of the European 
Communities in 1976~\citep[chap. 12.1]{hutchins1986machine}.

Research and commercial development in Machine Translation continued in the paradigm of Rule-Based Machine 
Translation (\GLS{RBMT}), in which a dictionary, a set of grammatical rules, and varying degrees of 
linguistic annotation are used to produce a translation, until the early 1990s, when a group of IBM 
researchers developed the first ``Statistical Machine Translation'' system, 
Candide~\citep{berger1994candide}. Building on earlier successes in Automatic Speech Recognition, which 
applied Shannon's Information Theory, the group used the availability of a bilingual corpus -- the 
Canadian Hansard -- to apply similar techniques to the task of French-English translation. 

In place of dictionaries and rules, statistical MT uses word alignments learned from a 
corpus~\citep{brown1993}: given a set of sentences that are translations of each other, translations 
of words are learned based on their co-occurrence (the \textit{translation model}); of the possible 
translations, the most likely is chosen, based on context (the \textit{language model}).

``Phrase-based MT''~\citep{koehn2003} is an extension of statistical MT that extends word-based 
translation to ``phrases''\footnote{That is: contiguous chunks of collocated words, rather than a 
``phrase'' in the linguistic sense.}, which better capture differences between languages. Although 
attempts have been made to include linguistic information~\citep[e.g.][]{koehn2007}, phrase-based MT 
was, until quite recently, the dominant paradigm in Machine Translation, and is still in wide use in 
industry. 

Google started providing an online translation service in 2006, initially using SYSTRAN's rule-based 
system, but switched to a proprietary phrase-based system in 2007~\citep{tyson2012}. More recently,
Google has started to make advances in Neural MT~\citep{wu2016gnmt}, which it has begun to use in
production\footnote{\href{https://www.blog.google/products/translate/found-translation-more-accurate-fluent-sentences-google-translate/}{https://www.blog.google/products/translate/found-translation-more-accurate-fluent-sentences-google-translate/}}.

\section{Apertium}
\label{sect:bgapertium}

Apertium is a shallow-transfer rule-based open source Machine Translation platform~\citep{forcada11a}.
Originally designed for the romance languages of Spain, it has been extended to
support more divergent language pairs, such as Basque to English~\citep{oregan13peeking}.

Finite-state transducers are employed in almost all components of the system, with data specified in XML-based 
formats~\citep{apertium}, with most compiled to efficient binary representations using an algorithm for fast 
construction of transducers~\citep{rojas2005}. A \textit{left to right, longest match} strategy is employed 
by the morphological analyser -- using ``tokenize-as-you-analyse''~\citet{garrido-alenda02}\footnote{\citet[Ch. 3]{apertium} provide a textual 
description of the algorithm, as adapted to Apertium}, based on the contents of the dictionary -- and by the chunker. 

Apertium is designed to function as a set of Unix pipeline components: the system is composed of modular
programs, each accepting input and generating output using variants of the same ``text stream'' 
format~(see \ref{ssect:textstream}), where the variant depends on the purpose of the tool. Additional
components can be added to the system, depending on the needs of the language pair: for example, in the
very closely related language pair Spanish--Catalan, only a single lexical transfer component is
employed, while in the English--Spanish language pair, three stages of transfer are employed. External
components can be added to the pipeline, provided that they can parse the text stream format: for example,
a number of language pairs employ the external Constraint Grammar system for disambiguation.

A typical Apertium pipeline contains the following components:

\begin{itemize}
\item A \textit{deformatter}, to separate a documents markup (e.g., HTML or Microsoft Word's DocX) from 
the text itself
\item A \textit{morphological analyser}, which both lemmatises the surface forms of the words, which attaching 
a set of attributes to denote syntactic information, such as gender, case, and number
\item A \textit{part-of-speech tagger} (\GLS{POS} tagger), typically using hidden Markov models (\GLS{HMM}), 
which disambiguates the possible morphological analyses based on context, and ensuring only one analysis is selected
\item A \textit{pre-transfer component}, which splits fused words into their individual lemmata (based on the 
form selected by the POS tagger)
\item A \textit{structural transfer component}, which, depending on the configuration, may also perform lexical 
transfer. The structural transfer component is the only transfer component in simpler language pairs, or, in more 
complex language pairs, acts as a ``chunker'' (see~\ref{ssect:apertiumchunk}), generating zero or more target 
language output chunks based on fixed patterns of input from the source language. In either case, it may perform 
local agreement and reordering operations, and can insert or delete words based on context.
\item An \textit{interchunk component}, which, in the chunking configuration, performs broader reordering and 
agreement operations. The chunk itself resembles the token received by the structural transfer component as input, 
and can have its own lemma and tags, typically set from its content, with the additional of an immutable content 
part, which contains the words associated with the chunk.
\item A \textit{postchunk component}, which extracts the words from a chunk, applying further agreement operations 
on the basis of the words within the chunk, as well as the tags of the chunk itself.
\item A \textit{morphological generator}, typically created from the same source as the morphological analyser, which 
outputs the surface form on the basis of lemma and tags.
\item An \textit{orthographical correction component}, which performs context-sensitive orthographical operations on 
the output of the morphological generator, such as contractions. 
\end{itemize}

\subsection{Chunking in Apertium}
\label{ssect:apertiumchunk}

Structural transfer in Apertium is based on fixed-length patterns. As these
patterns typically correspond to linguistic chunks, and because in its 
multi-stage configuration, the lexical transfer component groups its output
into chunks, this process is referred to as chunking; it is not, however,
the same process as is typically intended in natural language processing, where
chunking is a part of the analysis process. In Apertium, chunking is 
translation-driven, and the ``chunks'' it processes do not always have a
one-to-one correspondence with the chunks of a single-language analysis.

One common pattern that frequently occurs among European languages where the
chunking boundaries are blurred (for the purpose of translation) is where verbs
and pronouns collocate. For example: although French is, like English, an SVO
language, in sentences containing personal pronouns as the object, it adopts
an SOV configuration. Similarly, between Irish and English there is a need to
consider object pronouns together with the verb when translating the English
continuous tenses: ``doing it'' ought not to be translated as ``ag d\'eanamh \'e'',
but rather as ``\'a dh\'eanamh''; that is, the two English chunks need to be
merged into a single Irish chunk (and vice-versa in the other translation
direction).

This consideration of target language aside, as chunking is performed by the
structural transfer component, it takes care of local agreements and reordering.
As later processing stages which move chunks are based around this initial
structural transfer component, they inherit from it a structure based on the
morphologically analysed tokens that the initial transfer component receives;
that is, the chunks it outputs are not only assigned chunk tags to reflect
the type of phrase contained within the chunk, but are assigned a chunk ``lemma'',
which more accurately describes its contents, as well as any tags that are
deemed relevant. So, in a chunk containing a noun phrase, the contents of the
tags of the noun that describe its gender and number can be propagated to the
chunk, which the later stages can propagate to other chunks, to ensure, for 
example, that the verb agrees with its subject. The tags assigned to words within the chunk can
be replaced with a pointer to the chunk's own tags, meaning that changes to
those chunks are passed to those words immediately.

A final phase, post-chunk, removes the chunking markers, outputting the words within. Tags
that are linked to those of the chunk are replaced with its values, and further
processing can be performed.


\subsection{Apertium's text stream}
\label{ssect:textstream}

The various components of Apertium operate on two fundamental types of input:
tokens, and blanks. The first component of the Apertium pipeline serves to
normalise formatting details, by encapsulating them in so-called ``superblanks'':
that is, the formatting information is enclosed within square brackets, whose 
contents are treated by later components as though they were simple spaces.

As an example, the HTML fragment \texttt{the <b>third</b> man}, where ``third''
has been marked up with the HTML \texttt{<b>} element, to cause the text to
be rendered in boldface, would be ``deformatted'' by Apertium's HTML component
as \texttt{the[ <b>]third[<\/b> ]man}.

Tokens, the other fundamental type, can differ depending on the stage of pipeline.
The common elements of each type of token are the token boundary characters:~\texttt{\^{}}
to mark the start of the token, and \texttt{\$} to mark its end; \texttt{/} is used as a
separator in the event of multiple possibilities, \texttt{+} to mark a word internal split,
and \texttt{<} and \texttt{>} are used to enclose individual tags. \texttt{\#} is used
to mark an invariant ``queue'' of items in a multiword such as ``passer by'' where
``passer'' inflects, but ``by'' does not.

The output of morphological analysis is a token that consists of the surface form, followed
by a set of possible analyses. The output of Apertium's morphological analyser for the 
ambiguous English word ``open'' is \\ \texttt{\^{}open/open<adj>/open<vblex><inf>/open<vblex><pres>\$}:
the surface form ``open'' is followed by three analyses: adjective, infinitive, and present tense.
The Spanish contraction ``del'' is analysed as \texttt{\^{}del/de<pr>+el<det><def><m><sg>\$}, where
the preposition ``de'' is cut from the definite article ``el'', and the multiword ``speed of light''
is analysed as \texttt{\^{}speed of light/speed<n><sg>\# of light\$}, where the inflecting noun
``speed'' is followed by the invariant ``of light''\footnote{There are various speeds of light,
depending on the medium through which it passes; the one we  most frequently hear about is the speed 
of light in a vacuum.}

Apertium's tagger discards the surface form, and selects the single most likely candidate word,
based on the context of the previous word. For example, if, in the first example, the adjective
``open'' had been selected as the most likely, the output would look like \texttt{\^{}open<adj>\$}.
This form of token is both the input to the structural transfer module, as well as its output,
when not run as a chunker. When run as a chunker, the output of the fragment ``big dog''
\texttt{\^{}Nom\_adj<SN><UNDET><GD><sg>\{\^{}perro<n><3><4>\$ \^{}grande<adj><mf><4>\$\}\$}


\section{Statistical Machine Translation}
\label{sect:bgsmt}

\subsection{Overview}

At their most basic, Statistical Machine Translation systems use a pair of probabilities: the probabilty of the translation
of a word, given the source word, and the probability of the translation in context. The probabilities of the
translations are learned from a corpus of bilingual sentence pairs, in a process that's usually likened
to using a menu from a Chinese restaurant to find the Chinese character for some item, such as soup.
Figure~\ref{fig:chinesemenu} shows a sample from a Chinese menu. The only word common to the English items
is ``soup'', while the only character common to all three Chinese entries is ``汤'', making it the most
probably translation of ``soup''. Similarly, ``鱼'' appears twice, with ``carp soup'', and ``fish head soup'';
although ``fish'' is absent from ``carp soup'', if one knows that carp is a type of fish, then one can conclude
that this character means ``fish''.

With a larger corpus of menu items, it might be possible to determine that in addition to ``soup'', the character
``汤'' also collocates with ``tea'', giving us two possible translations. The language model is used to select
the most likely of these in context: ``fish soup'' is a fair more likely word pairing in English than ``fish tea'',
while ``green tea'' is similarly far more likely than ``green soup''.

\begin{figure*}
\begin{center}
\begin{tabular}{|l|l|}
\hline
Chinese & English \\
\hline
乌鸡\textbf{汤} & Black Chicken \textbf{Soup} \\
鲫鱼\textbf{汤} & Carp \textbf{Soup} \\
鱼头\textbf{汤} & Fish Head \textbf{Soup} \\
\hline
\end{tabular}
\end{center}
\caption{Chinese menu excerpt}
\label{fig:chinesemenu}
\end{figure*}


Rather than individual words, most SMT systems use ``phrases'', or, rather, n-grams, as the basic unit
of translation. An n-gram is a contiguous group of tokens--typically words, though punctuation 
and other such items may be considered--of size \textit{n}: a \textit{unigram} is a single token,
a \textit{bigram} is two, and a \textit{trigram} is three. Numbers higher than three are typically
referred to by number: \textit{4-gram}, \textit{5-gram}, etc. Returning to the menu example, simply
assuming a one-to-one, left-to-right correspondence between Chinese characters and English words
would be sufficient for the first and third examples, but it can be assumed that both Chinese 
characters ``鲫鱼'' are required to translate ``carp''\footnote{It seems to mean ``crucian carp'' specifically.}.
 
In its broadest sense, a language model can be any model of a language.
For example, a simple list of individual words, such as those used as a source of correct spellings
by a spell checker, can be considered a language model. In Statistical Machine Translation (as well as ASR), the term
\textit{language model} more typically refers to an n-gram language model, with corpus-derived 
probabilities for each n-gram. For such a language model of size \textit{n}, the model will contain
all sizes of n-gram from 1 to \textit{n}.
 
N-grams, in the form of lexical bundles~\citep{biber2004bundlers}, have shown increasing use in Linguistics; in
Applied Linguistics in particular. For example, a former colleague's dissertation, in contrasting the academic
writings of native speakers with those of Polish native speakers, found that while, as expected, there was
a significantly higher reliance on formulaic language; however, when the most frequently used bundle (``on the other hand'')
was excluded, the non-natives showed significantly less use of bundles, indicating that English language learners
could better served by an increased focus on formulaic language~\citep{kozicka2016dissertation}.

The essential property of n-grams as a unit is that they overlap: the probability of a whole sentence can be
estimated using just a small window of n-grams. This property is used within n-gram language models to
allow rare n-grams to be estimated using ``backoff'': if, in a trigram model, there is no evidence of a
particular trigram, the probability can be estimated by using the probabilities of the two bigrams it is
composed of.


The development of the first Statistical Machine Translation, CANDIDE~\citep{berger1994candide}, 
has its roots in earlier developments at IBM in Automatic Speech Recognition (\GLS{ASR})~\citep{Jelinek2009dawn}.
In statistical ASR, the probability of a sentence $w$ is given in terms of its utterances, $U$, as 
in~\ref{eqn:asr1}~~\citep[p. 538, (3)]{jelinek1976asr}.

\begin{equation}
\label{eqn:asr1}
P\{w|U\}=P\{U|w\}P\{w\}/P\{U\}
\end{equation}

This equation, when stripped of the independent $P\{U\}$, as reframed in terms of $A$, the acoustic
signal, rather than utterances, is given in~\ref{eqn:asr2}~\citep[p. 487]{Jelinek2009dawn}, where
$\hat{P}(A|W)$, the probability of word $W$ resulting from the acoustic signal $A$, is called the
\textbf{acoustic model}, and $\hat{P}(W)$, the probability of word $W$, is called the \textbf{language model}.
The combination of both is known as the \textbf{noisy channel model}~\citep[pp. 95--96]{koehn2010statistical},
originally proposed in work on error-correcting codes.~\citep{6773024}.

\begin{equation}
\label{eqn:asr2}
\mathbf{\hat{W}} = \mathrm{arg} \underset{\mathbf{W} \in \mathcal{S}}{\mathrm{max}} \Pr{(\mathbf{W}|\mathbf{A})} =  \mathrm{arg} \underset{\mathbf{W} \in \mathcal{S}}{\mathrm{max}} \Pr{(\mathbf{A}|\mathbf{W})} \Pr{(\mathbf{W})}
\end{equation}

Following the success of the statistical approach to ASR, the team at IBM applied their methods to Machine 
Translation, using a similar formula: \ref{eqn:smt1}~\citep[p. 492]{Jelinek2009dawn}, where \textit{E} means English, and \textit{F} French;
this pair was chosen because of the similarity of the languages, and because of the availability of a parallel corpus,
the Canadian Hansard~\citep[p. 493]{Jelinek2009dawn}.

\begin{equation}
\label{eqn:smt1}
\mathbf{\hat{E}} = \mathrm{arg} \underset{E}{\mathrm{max}} \Pr{(\mathbf{F}|\mathbf{E})} \Pr{(\mathbf{E})}
\end{equation}

The translation model is obtained from a corpus through a process called \textit{word alignment}~\citep{brown1993}, 
using the Expectation Maximisation algorithm: first, the model is initialised, often using uniform distributions;
second, the expectation step, the model is applied to the data; third, the maximisation step, the model is learned
from the data; and fourth, steps two and three are repeated until convergence. \citet{brown1993} describe a series
of five translation models, where, aside from the first, which is initialised uniformly, the first step of the EM
algorithm is performed using the output of the previous model. Although these models take into consideration a
possible ``null'' alignment, where a word is inserted or deleted, they cannot generate many-to-many 
alignments~\citep[p. 116]{koehn2010statistical}. To overcome this, alignments are computed in both directions, and
merged by ``symmetrisation of alignments''~\citep{och2004alignment}.

Phrase-based SMT extends word alignment to ``phrases'', or, rather, n-grams, by extracting heuristically minimal n-gram 
pairs, based on a word alignment~\citep{och2004alignment}. The phrases are generalised using bilingual word 
classes~\citep{och1999wordclass} to form ``alignment templates''~\citep{och2004alignment}, which allow a greater
amount of phrases to be extracted from the corpus, providing more context than could be learned using word
alignment alone.

Factored translation models~\citep{koehn2007} extend the phrase-based model with additional annotation -- typically, 
linguistic annotation, such as part-of-speech tags or lemma -- as a means of better generalising the translation by
adding the sort of information that allows knowledge-driven approaches to degrade better than other data-driven
approaches. Despite the positive results reported on factored models, however, what little gain there may be is
typically outweighed by the extra data required -- one extra factor effectively doubles the data used; a second, 
triples it -- and the slowdown of the translation process this entails. Additionally, it has been shown that
in contrast with the use of factors in neural machine translation, factors in phrase based SMT generalise poorly, as
``the individual models either treat each feature combination
as an atomic unit, resulting in data sparsity, or assume 
independence between features, for instance
by having separate language models for words and
POS tags''~\cite[p. 89]{sennrich-haddow:2016:WMT}.

\citet{Chiang:2005:HPM:1219840.1219873} extended the phrase-based model to ``hierarchical phrases'', learning a
synchronous context-free grammar from bilingual texts -- in effect, learning something equivalent to a pair
of constituent parsers, but without syntactic constituent labels, to allow the use of embedded phrases in 
phrase-based SMT~\citep[e.g.,][]{Zollmann:2006:SAM:1654650.1654671}, which have been shown to be of particular
benefit to language pairs where a lot of reordering is required~\citep{Zollmann:2008:SCP:1599081.1599225}.

The influence of ASR on Statistical Machine Translation did not end with CANDIDE; following work on combining
all of the components of an ASR system as a single weighted finite state transducer~\citep{MOHRI200269}, which
quickly became the default paradigm, thanks to toolkits such as Kaldi~\citep{Povey_ASRU2011_2011}, work 
soon appeared on applying similar techniques to phrase-based SMT~\citep{Iglesias:2009:HPT:1620754.1620817},
followed, thanks to the application of pushdown transducers~\citep{Allauzen:2012:PTE:2402069.2402077} (an 
extension of automata that adds a stack, extending their expressivity to Context-Free Grammars), by 
hierarchical SMT~\citep{Iglesias:2011:HPT:2145432.2145577}.

\section{Hybrid Machine Translation}

There are a number of methods that have been used to try to combine the best of two or more approaches to
Machine Translation. One of the more common methods is the ``multi engine'' approach, where the outputs
of two or more systems are combined to yield a better translation~\citep{Heafield-wmt10,Hogan:1998:EMM:648179.749224,eisele2008hybridmemt},
using statistical methods to combine the output.

Another method is to use statistical methods to generate rules~\citep{dugast2009selective} and/or lexica~\citep{Surcin_2007.rapid} for knowledge-driven systems.
For Apertium specifically, there are a number of open source tools available that perform this task of 
inducing rules from a parallel corpus, using methods employed by phrase-based SMT~\citep{sanchez06a,sanchez-cartagena16b},
as well as a tool that can extract lexical data~\citep{Caseli2006}.

The opposite method is also used, to enhance phrase-based systems with the output of RBMT systems; \citet{sanchez-cartagena12a}
provide an open source toolkit to augment the Moses SMT toolkit~\citep{Koehn:2007:MOS:1557769.1557821} with phrases generated
using Apertium. This approach has been taken to its extreme by using an SMT system to try to relearn an RBMT system~\citep{Dugast:2008:WRR:1626394.1626421}.

A final commonly used method of hybridisation is the use of SMT as a postcorrection system for the output of
an RBMT system: instead of directly aligning the source and target languages, the source language is first
translated using the RBMT system, with its output used as the source language to the SMT system, to learn
a set of corrections~\citep{dugast-senellart-koehn:2007:WMT}. Similarly, the opposite has also been attempted,
using an RBMT system for long-distance reordering and grammatical generation of SMT-generated input~\citep{ahsan2010coupling}.

\section{Neural Machine Translation}

The earliest neural network, the McCulloch-Pitt Neuron~\citep{McCulloch1988}, was modeled on brain function.
It used a linear classification model, with weights that could be set by its operator. The 
perceptron~\citep{Rosenblatt58theperceptron} was the first model that could
learn weights given examples of inputs~\citep[p. 15]{goodfellow2016dl}.

Neural networks have come in and out of fashion several times since the 1950s, with the most recent
return to fashion, in the guise of ``deep learning'', made possible by a number of factors, in 
particular the availability of methods to train networks with multiple layers, and access to high-powered
computing, thanks to Graphics Processing Units (\GLS{GPU}), which (as the name suggests) were originally
designed for accelerating graphical operations, such as those used in computer games, but thanks to their
number-crunching power, have been adapted to other tasks.

The first headline-grabbing success story of this current resurgence of neural networks was in the area
of computer vision, when a neural network based object detection system, trained using GPUs, significantly
outperformed the previous state of the art~\citep{NIPS2012_4824}.

In the area of Natural Language Processing, one of the first areas where neural networks had a big impact
was in language modeling~\citep{mikolov2012phd}, thanks, in part, to the broad availability of an open source
implementation. The headline-grabber in NLP, however, was word2vec~\citep{DBLP:journals/corr/abs-1301-3781}.

Although itself based on a relatively shallow, 2 layer neural network, the word2vec method of creating word
embeddings provides a distributed representation of words suitable for use in deeper neural architectures. Designed for the task of finding analogies among
words, the vector space positions occupied the word vectors are organised so that common contexts are close
in proximity. Given a seed lexicon of bilingual mappings, this has been used across languages to find
translations based on cross lingual projection of these shared contexts~\citep{44931}.

A recurrent neural network is a type of neural network that operates on a sequence: at each time state $t$,
of sequence $x = (x_1, \ldots x_t)$, the hidden state $h_t = f(W_{h}h_{t - 1} + W_{x}x_{t})$; it is calculated on the basis of the previous time step $h_{t - 1}$ multiplied by its weight matrix, $W_h$, added to the input at the same time step, $x_t$, multiplied by the weight matrix $W_x$. The output $o_t$ is calculated as $o_t = softmax(W_{o}h_t)$. The weight matrices are shared, so the same calculations are performed at each time step, but with different inputs.

Recurrent neural networks have limits to what they can remember; variants such as Long Short-Term Memory (LSTM)~\citep{Hochreiter:1997:LSM:1246443.1246450} and Gated Recurrent Unit (GRU)~\citep{cho2014rnnencdec} add gate mechanisms that are used to decide what to remember and what to forget. 

\citet{cho2014rnnencdec} use a pair of recurrent neural networks for Neural Machine Translation. One, the encoder, transforms the input to a fixed size vector from the target language, while the other, the decoder, generates the output from this vector; the encoder and decoder are jointly trained to maximise the probability of the output given the source.
% chap3.tex (Definitions and Theorem)

\chapter{Development}
\label{chap:devel}
%\epigraphfontsize{\small\itshape}
\epigraph{\textit{\small{I promised my right hemisphere it was never going to have to work full time}}}{``Oliwka N.''}

\section{Rule-Based Machine Translation}
\label{sect:ch3rbmt}

In this section, the development of the RBMT system will be described, covering the resources used to create the system, 
tools created to convert those resources and to more efficiently employ them, as well as outlining difficulties encountered.

The creation of the RBMT system was part of an internship, which came with two primary
requirements: the conversion of the bilingual data (see~\ref{ssect:devrbmtbil}) and the
creation of a set of transfer rules.

To this initial set of requirements were added the additional tasks of converting 
IrishFST~\citep{elain09} for the analysis and generation of Irish, and the creation
of a tool to allow the specification of rules in a manner that either allowed or
approximated recursion in the rules.

\subsection{Bilingual data}
\label{ssect:devrbmtbil}

The bilingual data used for the RBMT system is derived from data provided
by The Department of Culture, Heritage and the Gaeltacht, consisting of
English-Irish Dictionary (EID)~\citep{debhaldraithe1959english} and Focl{\'o}ir 
Gaeilge-B{\'e}arla (FGB)~\citep{donaill1977focloir}, from which permission was
granted to create an open source lexicon to contribute to the Apertium
project.

\subsubsection{Dictionary cleaning}

The only source of information available on the XML editions of the dictionaries
is from a blog post\footnote{\href{https://multikulti.wordpress.com/2014/01/04/how-to-retro-digitize-a-dictionary/}{https://multikulti.wordpress.com/2014/01/04/how-to-retro-digitize-a-dictionary/}} 
that describes the process of creating the XML from the source data.

The blog post mentions that EID had been scanned, OCRed, proof-read and marked
up into XML, but that the XML structure had been lost during the proofreading
process, and that attempts were made to infer the structure based primarily
on the formatting that remained. FGB, on the other hand, was generated from
the original typesetting files.

%FIXME: Kevin did the OCR, mentioned that 30\% of the source words are not real words, etc.

An initial extraction of the simplest (and most common) types of entry layout,
based on a na\:ive regular expression-based extraction,
had a quite high yield: 13076 entries, but missed out on some quite
common words: for example, the entry for ``girl'' contains two subsenses, and 
a number of empty sub-subsense entries (which in the original are explained by
example, rather than by direct translation). The simplistic parser was unable 
to account for these structures, and I was tasked with parsing the XML in a more
robust manner.

This loss of structure has proven to a quite difficult obstacle, and the first two
attempts at parsing the XML directly lead to failure. The first attempt used
a Java-based parser that attempted to read the sub-elements of each entry in a
linear fashion, but attempting to compensate for some of the idiosyncracies in
the XML was ultimately unsuccessful. A second attempt used Scala, and leveraged
Scala's pattern matching facilities to emulate the regular expression-based
extraction for simpler entries, with a fall-back to the serial approach used
in the Java parser for complicated entries that did not match those simpler
entries\footnote{As Scala runs on the JVM, Java-based classes are essentially 
native to it.}. An initial attempt was made to make corrections to the XML structure
while parsing, but this swiftly became over-complicated and unmanageable.

While attempting to correct the structure of the XML, I came across some 
erroneous components of the entries that were also present in the online
version of the dictionaries; I therefore decided to revisit my strategy for
parsing the XML, by making it a layered process.

The first layer addresses mistakes in the XML in a simple perl script that
uses regular expression-based transformations to the text before it is parsed.
As well as simplifying the XML input, separating this stage of the parsing
has the added advantage of being packaged in a way that can be easily passed 
back to the maintainers of the online version.

The scripts handle a number of different issues in the source, from errors to minor 
inconsistencies. The difference in the structure of both dictionaries is perhaps 
reflected in the number of transformations performed by each: 193 transformations 
in EID, compared to 76 in FGB.

Taking into account that EID was scanned, OCRed, and proofread, there are surprisingly 
few OCR errors: at the time of writing, I have found only one such error\footnote{There 
is also ``clo dubh'' given as translation for ``black-letter'', instead of ``cl\'o dubh''; 
and ``duct aeir'' rather than ``ducht aeir'', but as I have not verified these entries 
against the printed edition I cannot tell if these are transcription errors.}, where 
``hurling'' was translated as ``lom\'ana\'iocht'' instead of ``Iom\'ana\'iocht'': that 
is, with a lowercase ``L'' in place of an uppercase ``i''. 

Most of the translations are aimed at consistency: for example, where a translation has
 been mislabeled as a categorical label, or in merging separate translation elements that 
would otherwise lead to incorrect translation entries being generated (``etc.'' is a 
frequent offender).

The current version of the parser operates on the transformed XML. As well as having 
more consistent input than the previous attempts, it no longer attempts to parse the 
entries in a purely linear fashion: for example, the element \texttt{$<$b$>$} in the 
XML edition of FGB serves two purposes: when following $<$g$>$ it contains a word form, 
or information from which to construct one.
In other contexts, it can contain a subsense number, preceding a 
translation of that other sense of the word.

In both dictionaries, this re-use of elements is a frequent source of errors: one 
particular example is of the \texttt{$<$r$>$} element in FGB. In a ``see also'' context,
it signifies that a Roman numeral is part of the entry reference; when contained by a
\texttt{$<$trans$>$} element, it contains the translation(s); and elsewhere it is used
to contain usage information in parentheses. One frequently occurring problem in FGB is that some
pieces that ought to have been marked as translations have not been, presumably because 
the translation began with a parenthetical piece (most commonly ``(act of)'').

The current parser, upon reading a $<$g$>$ element, checks if the next is $<$b$>$, and if so, 
it is consumed by the $<$g$>$ element. As well as simplifying the parsing process, this ensures 
that the grammatical forms are kept with the information about those forms.

\subsubsection{Entry replacement}

As Irish has no single citation form for verbs, it is common to either use the first person singular 
present tense form (``t\'aim'', ``scr\'iobhaim'') or the second person imperative singular (``b\'i'', 
``scr\'iobh'') as the citation form.

It was clear from the outset that the translations of verbs in EID would need to be replaced, primarily 
for consistency, as it uses the first person form of the verb, while both FGB and the morphological 
analyser use the imperative. These replacements were quite simple to generate, using the analyser, but 
the generation process uncovered a more pressing reason for replacing verb translations: the ``compression'' 
of shared elements in verb phrases. For example, the entry ``yowl\textsuperscript{2}'' contains the translation 
``Ligeann, casann, uaill'', where the common element of the verb phrases ``ligeann uaill'' and ``casann uaill'' 
is shared among them, for conciseness.

This changed the replacement process into a two-stage process: first, the whole translation text is 
compared against a list of replacements for multiwords of this kind, which additionally contains pattern 
rule information, to generate a single entry from the multiword; if the entry is missing from this list, 
it is split by comma and semi-colon, and each individual word is checked against the original verb form 
mapping list. If the word is absent from this list, it is marked to be ignored: an output entry is created, 
but flagged so that it will not be used for translation -- words that cannot be generated by the analyser 
cannot fully be utilised by the translator, as they cannot be inflected.

Later private correspondence with Kevin Scannell\footnote{A noted developer of language technologies for 
Irish and other Celtic and minority languages} revealed not only that he had OCRed EID, and created the 
XML structure that had been lost, but his later work with a Manx translation of it had caused him to delve 
a little deeper into the history of EID, which had, in turn, been based on an earlier French-English 
dictionary. He warned that 30\% of the English headwords make no appearance in a corpus of contemporary English.

Taking this warning on board, I checked first the list of words that contain a hyphen in the name, some of 
which (such as ``black-letter'' and ``teen-ager'') had a distinctly archaic appearance. This lead to the 
addition of a similar replacement mechanism for headwords; but, as FGB shared at least some history with EID, 
I checked the translations of some of the hyphenated words against the list of words to be replaced in EID, 
and found the same need for replacement there. Fortunately, the high degree of overlap meant that the same 
replacement list could perform double duty. The list itself was, for the most part, automatically generated, 
by generating both a single word and separated words from the hyphenated form, and checking these against 
external lexical sources for English.

Extending my inspection of translations beyond verbs, I found that a similar need for 
replacement existed in other categories. For the most part, this needed nothing more 
than the addition of the lists of multiword generation patterns that were already 
required to treat the multiwords as a single unit, but where a whole translation needed 
to be replaced, this required something extra in EID, because of the presence of 
grammatical elements to mark gender.

The recognition that these replacement mechanisms formed a basis for addressing language 
change, by replacing the contents with something more relevant to the living languages, 
lead to a further replacement stage, to precede the others: at the time the entry title 
and its primary sense key are read, another list is consulted, from which a replacement 
for the entire entry might be taken. 
For example, figure~\ref{fig:commission2} contains the definition of the verb 
``commission'' from EID; in modern use, all of these senses have been replaced 
by the verb ``coimisi\'unaigh'', which does not appear in either dictionary. 
This entry replacement mechanism allows the whole entry to be replaced by this 
single translation.

This mechanism allows the insertion of a different headword, where appropriate: 
``ruaille\textsuperscript{2}'' in FGB exists to stand for ``ruaille buaille''. 
As this is treated as a phrase, the XML entry contains no useful information. 
With this mechanism, ``ruaille buaille'' can be inserted in place of ``ruaille''.

\begin{figure}
\textbf{commission}\textsuperscript{2}, \textit{v.tr.} 1 \textbf{a} Tugaim údarás do; údaraím. \textbf{b} Ainmním (duine) ina oifigeach; tugaim coimisiún do (dhuine). \textbf{c} Ordaím, iarraim (leabhar a scríobh, etc.). 2 \textit{Nau:} Armálaim, feistím (long).
\label{fig:commission2}
\caption{Definition of commission\textsuperscript{2} from EID.}
\end{figure}

\subsubsection{Current status}

The latest extraction produced 88243 distinct entries. Although there is significant 
overlap with the previous 13076 entries, it is not a complete overlap: the differences 
need to be inspected, to ensure that the differences are not due to regressions 
introduced in the new code.

\subsection{IrishFST conversion}

Translators created using Apertium use, for the most part, a common set of ``tags'' to
describe linguistic features: \texttt{n} for nouns, \texttt{vblex} for (lexical) verbs, etc.
This is in part due to the fact that the same small group of people have been involved
in the development of the majority of language pairs, but primarily to simplify the
transfer process: the same attributes will, by default, be copied from source to target,
so features such as \texttt{sg} (singular) and \texttt{pl} (plural) do not need to
have any specific action added in the default case. To take advantage of this, the 
tagset used in IrishFST needed to be adapted.

The first pass at performing this adaptation went quite easily, aside from a few
corner cases where a pair of tags in one corresponded with a single tag in the other.
More difficult, however, was the conversion of the lexicon to Apertium's conventions.

Apertium uses a simple concatenative model of inflectional paradigms, where the common
substring of the word has a paradigm reference added to it, from which the various forms
of the word are expanded. IrishFST uses a more complex system where the continuation class
of a word is added in its entry, from which the individual forms are computed, based on
the word's stem.

An initial set of conversions was performed using an Apertium tool for generating paradigms
for an expanded list of forms; this tool, however, can only accommodate suffix inflection.

One other difference between the systems is that Apertium's lexical processing unit, thanks
to the use of ``tokenize-as-you-analyse'', is able to process multiword units as words-with-spaces;
it also has a facility for breaking apart contracted words; to simplify the transfer of Irish
conjugated prepositions, such as ``liom'' (with me), I opted to convert these prepositions to
follow Apertium's conventions, which allows conjugated prepositions to be treated the same
as unconjugated prepositions, deferring their combination to the orthographic post-processing
module. This also allows a more satisfactory treatment of genuine contractions in IrishFST,
such as ``s\'eard'', a contraction of ``is \'e an rud'' (\textit{that's the thing}), which
can be split into its components, rather than, as in IrishFST, being tagged with the features
of each, but with only the lemma of the first.

The resulting converter, in comparison with the original conversion script, is quite complex.
It performs the same paradigm generation as the existing Apertium tool, but takes into
account Irish prefix mutations. This facility also allowed the pinpointing of several errors
in IrishFST, where a number of prefixed forms were missing tags to represent the mutation.

\subsection{Engine modifications}

The needs of Irish required some changes to Apertium's lexical processing tools. Unmodified,
Apertium's morphological generation component was unable to generate the correct form when
a word beginning with a vowel was prefixed with ``n'' or ``t''; attempting to add separate
entries for both upper and lower case word forms lead to two outputs: one correct, as well
as the original incorrect form. I modified Apertium's lexical processor to add a mode that
allows forms specified in this way to have a unique, correctly generated form.

In addition to this, because of the nature of much of the publicly available bilingual data
for Irish (see \ref{ssect:smtdata}), I extended the lexical processor to allow it to be
provided with a list of characters to be ignored in processing, intended to discard some
of the non-printing characters in Unicode, such as soft hyphen -- a character which is
normally invisible, used to provide a place to hyphenate a word, to split it across lines in
justified text.

Although not strictly necessary, the modifications that made ignoring specified characters
possible were similar to the changes that would be necessary for diacritic restoration; as
this was now a relatively simple change, I added it, as quite a lot of publicly available
Irish text is written without the use of fada (the acute accent).

All of these modifications have been accepted into the public version control system of
Apertium, and will be included in the next public release of the software.

\subsection{Rule conversion}

To meet the secondary requirement of a recursive, or quasi-recursive, means of specifying
transfer rules, I used \texttt{apertium-transfer-tools}~\citep{sanchez06a} as a model (and,
to a lesser extent, its successor, ruLearn~\citep{sanchez-cartagena16b}): this set of tools
are designed to create Apertium rules from a parallel corpus, using techniques from Statistical
Machine Translation.

The basic format of such rules -- a set of words in both language, and their alignments -- seemed
vastly preferable to the extremely verbose XML rules used by Apertium. The first limitation of
this toolset is that it only generates rules appropriate for a single level of transfer, while
it was quite apparent that for Irish-English, chunking would be necessary. ruLearn attempts
to augment the original transfer tools package with a means of generating chunks, but it can
only do so when there is an Apertium chunker already available, which is not the case for Irish.

Adding to this, there are some significant mismatches between chunks in English, and those in
Irish. One example is verb phrases in the continuous aspect: ``be going'' is (arguably) a single
chunk in English, but the Irish ``b\'i ag dul'' must be two, to allow the subject to be placed
after the finite verb ``b\'i'', and before the verbal noun phrase ``ag dul''. 

Another complication is that, in contrast with a language such as Spanish, where the base case of
most Apertium rules is typically quite regular, most chunking rules between English and Irish
will have more than one regular correspondence: in phrases containing a determiner, if it is
the English indefinite article, it must be omitted; if it is ``this'' or ``that'', the rest of
the noun phrase will have the definite article ``an'' as its leftmost component, and ``seo'' or
``sin'' as the rightmost; for a verb phrase, the preferred translation could be a verb, or, in
the case of a number of frequent verbs, the preferred Irish equivalent could be a copula phrase:
``like'' becomes ``is maith le'', ``prefer'' becomes ``is fearr le'', etc.

Apertium's transfer system allows common operations to be placed in macros, which can be shared
among rules, simplifying them greatly. One common macro operation necessary for English to Irish
is to add the type of a determiner to the chunk's tags: Irish genitive constructs have a definiteness
restriction, much like that of the Anglo-Saxon genitive in English: only one component may be definite;
at the level of an individual chunk, it cannot be known if the chunk is part of a genitive (or
prepositional) phrase, so the chunk is tagged for interchunk, where it can be known, and during
reordering, the tag changed as a signal to the postchunk processor to remove the article before
output. Similarly, where both sides of a phrase that \textit{seems} like a genitive use some
other definite determiner -- ``my photos of my holiday'' -- this chunk annotation is used to
replace the preposition with ``de'', instead of using a genitive construct.

Rule conversion, rather than replacement of the transfer mechanism, was selected because developing
a replacement transfer component was not feasible within the timeframe of the internship, and because
the prior experience of Apertium has shown that fixed-length patterns can be sufficient.

I became overly focused on finishing the rule converter. The first approach I took to recursive
expansion was overly complicated, trying to expand both sides of the rule at the same time, while
attempting to keep the positional indices used by the macros updated. In the week before the
deadline, I thought of a simpler approach, where each token contained both source and target sides
of the rule, and words inserted into the target side were appended; after expansion, these extra
parts are discarded from the source side, while on the target side the tokens are reordered,
according to their position.

While this part of the rule converter works, the more important part does not: as each multi-part
rule contains multiple versions of the same tokens to be expanded, which must be kept together
to maintain the rule as a whole, the expansion method for single-part rules simply cannot work.
The realisation came too late to recover from, and while work is ongoing to expand the rules
by hand, there is simply too much left to do to even attempt to run the system.

\section{Statistical Machine Translation}
\label{sect:ch3smt}

In this section, the details of training SMT systems will be described, including a description of the 
corpora employed. A particular point of interest in SMT are the efforts to incorporate linguistic information into the SMT 
process, such as factored models and hierarchical models, which seemed closest, at least in spirit, to
the knowledge-driven approach.

The purpose of investigating SMT was originally intended as a means of comparison
with the rule-based system, of acquiring data with which to expand it, and to investigate a possible 
method of hybridisation, where the rule-based component was to be used as a fall-back, almost in the
opposite manner to \citet{sanchezmartinez09d}: the input was to be segmented into light-weight, marker-based
chunks~\citep[e.g.,][]{gough2004robust}, but instead of inserting statistically-derived phrases into the output of the rule-based system, 
as in that work, each chunk would be checked to see if it was present in the statistical phrase table; 
if not, it would be passed to the rule-based system, as a fall-back, using the XML mechanism provided
by Moses to insert translation candidates from external sources.

This, however, was not to be: the reality of the data most commonly available -- and, as such, most likely
to be used in the training of SMT systems, generally -- was that it suffered quite severely from issues
of data quality.

\subsection{Data}
\label{ssect:smtdata}

\citet{ARCAN16.9} list a number of available sources of parallel text for English Irish MT (reproduced in figure~\ref{fig:smtcorpora}); however, I have concentrated
on using only parallel text available from the OPUS project~\citep{TIEDEMANN12.463}. Initially, this was simply as a test after
installing Moses; however, as OPUS is perhaps the only large scale effort to make available parallel texts in multiple languages,
and is both widely advertised -- for example, OpenNMT's FAQ\footnote{\href{http://opennmt.net/FAQ/}{http://opennmt.net/FAQ/}}
lists only OPUS as a source of parallel text -- and widely used, it seemed the most likely source of data
for anyone who would attempt to build an SMT system for English and Irish.

Of the subcorpora available as part of OPUS that contain text in Irish, two in particular stood out as the potential
sources of unusual translations from the baseline model: the EU Bookshop Corpus~\citep{SkadinsEA:LREC14} and the
KDE4 corpus~\citep[s. 2.3]{Tiedemann:RANLP5}.

In addition to the existing parallel data, I created a corpus by scraping the Citizen's Information website.
As I have not used it in my SMT experiments, I have not discussed it in this section, but as it may be a
useful resource, I have provided a brief description of its construction in appendix~\ref{app:citinfcorpus}.

\subsubsection{EU Bookshop Corpus}

The OPUS EU Bookshop Corpus was created by crawling PDF documents from the EU Bookshop website. 
As many of these documents are relatively old, a large amount of the text available in these PDF documents has been 
inserted using Optical Character Recognition (\GLS{OCR}); an additional complication is that, due to the nature of documents
produced by the EU, these documents can quite legitimately contain sections or phrases in other EU languages. Because of
this, the OCR system in use for these documents has been trained not on a single language, as is conventional, but on all
EU languages. In a conventional, single language OCR system, there are certain issues in accuracy due to visual similarity
of character combinations within the language: \texttt{m} and \texttt{in} can look similar, and can therefore be confused.

This is a widely known problem, and tools exist to tackle many of these issues, for single languages. The OCR text produced 
by the EU Bookshop, however, has the additional complication of multiple scripts, and even within a single script, there
is a confusion of accented letters: as well as, for example, the visual similarity of many letters of the Greek or Cyrillic
alphabets -- Greek Ρ for Latin P, etc. -- which existing tools do not take into consideration, there is also a confusion
of accented characters, with the acute accent (Irish fada) being replaced with umlaut, macron, and grave accents.

As well as the issue of OCR, the EU Bookshop Corpus contains a number of lines where text boundaries have been incorrectly 
detected in the process of conversion from PDF, with an automatic correction method employed to remove spaces from the
words~\citep[p. 1851]{SkadinsEA:LREC14}; this, however, does not take OCR errors into account, as in figure~\ref{fig:ocreubkshp}.

\begin{figure}
\begin{verbatim}
'Article l5l
If, within six weel < s of its being convened,the Conciliation 
Committee approves of th9 v9te1 aiäint text
\end{verbatim}
\label{fig:ocreubkshp}
\caption{Examples of OCR errors from the EU Bookshop corpus.}
\end{figure}

As the correction method employed does not account for such errors, it cannot
correct, e.g., `weel < s' to `weeks', nor can it correct the word boundaries of
the non-existant word. This issue of spacing is possibly the result
of wider spaces between words in justified text, which would also explain the existence of the opposite problem: missing
spaces between words~\citep[p. 1853]{SkadinsEA:LREC14}.

In testing a sample set of 200 sentences of English-Latvian~\citep[pp. 1852--1853]{SkadinsEA:LREC14}, it was found that
only 59.8\% of the sentences were considered ``good''. A set of automatic corrections were applied to that language
pair, with the resulting, much smaller, corpus achieving 85\% of sentences being considered ``good'' by human raters.

\subsubsection{KDE4 Corpus}

Unlike the EU Bookshop corpus, which was known to have quality issues due to its source data, the KDE4 corpus, which
originates in the localisation files used for programs in the KDE4 desktop environment, comes with no quality warnings.
This is not, however, to say that it does not contain quality issues, many of which seem to have been introduced by the 
conversion process. One example, visible in figure~\ref{fig:kdeenc}, is the mistaken recoding of texts from the 8-bit 
Latin1 encoding, to UTF-8. Also in that example, the segments have misaligned, possibly due to incorrect insertion on
the English side, or incorrect deletion on the Irish.

\begin{figure}
\begin{verbatim}
Shrink Icons	& Ollmhór
& Default Size	& An- mhór
& Huge	& Mór
& Very Large	& Measartha
& Large	& Beag
& Medium	An- bheag
& Small	Cumraigh an Cúlra...
\end{verbatim}
\label{fig:kdeenc}
\caption{Align of incorrectly converted and misaligned segments from KDE4.}
\end{figure}


Secondly, instructions to translators have been included, as shown in figure~\ref{fig:kdetrans}.

\begin{figure}
\begin{verbatim}
Your names	Kevin ScannellEMAIL OF TRANSLATORS
Your emails	kscanne at gmail dot com
January	Month
\end{verbatim}
\label{fig:kdetrans}
\caption{Instructions to translators misidentified as translations, from KDE4.}
\end{figure}

Finally, there seem to have been misaligned files in the corpus, as in figure~\ref{fig:kdemisalign}:
the translations on the English side seem to refer to a different program to those on the Irish.

\begin{figure}
\begin{verbatim}
Search Box	Sonraí Bibleagrafaíocha (BibTeX) Name
Newspaper activity	Sonraí Bibleagrafaíocha (BibTeXML) Name
\end{verbatim}
\label{fig:kdemisalign}
\caption{Misaligned segments from KDE4.}
\end{figure}

\subsubsection{Data cleaning}

\citet{GJ2008} present a method of automatically increasing the coverage of Translation Memory systems,
by allowing the specification of simple rules, using regular expressions, to automatically translate
items such as dates, times, etc. During an earlier phase of my internship, I adapted the Duckling
parser\footnote{\href{https://github.com/wit-ai/duckling\_old}{https://github.com/wit-ai/duckling\_old}} 
for Information Understanding to Irish: it parses similar items, converting them to a 
language-independent representation, which seemed an ideal way of implementing a similar system of
extending Translation Memory: the choice of programming language -- Clojure, a language that uses Java's JVM -- seemed a good fit with
existing open source Translation Memory systems, all of which (that I am aware of) are written in Java.
Although Duckling is able to parse such expressions, it is unable to generate them, and adding this
facility did not seem an appropriate use of time; additionally, Facebook (the developers of Duckling)
later rewrote it in a different, non-JVM language\footnote{Facebook converted my code, so the new version
retains support for Irish: \href{https://github.com/facebookincubator/duckling}{https://github.com/facebookincubator/duckling}}.

Faced with the various errors in the EU Bookshop and KDE4 corpora, I developed a proof-of-concept program
named Duckegg\footnote{Named both reflect the influence of Duckling, and as a nod to a former teacher, who
would refer to a zero in a test as ``a big fat duck egg'', in reference to the fact that, due to the relative 
haste in which it was written, the software will not win any design awards.} to perform a number of automatic
corrections.

Duckling inspired it in two ways: first, Duckling's pattern matching can use regular expressions, or can build
more complex patterns based on existing patterns; second, Duckling's patterns aim to be robust to informal writing.

Most of the proof-of-concept corrections implemented in Duckegg are specific to the gaois.ie corpus (referred
to as ``Irish legislation'' in figure~\ref{fig:smtcorpora}): one particular check is for Statutory Instrument
numbers, which frequently occur in the corpus. Although the English side of the corpus is quite clean\footnote{The
text is publicly available on \href{http://www.irishstatutebook.ie/}{http://www.irishstatutebook.ie/}}, the
Irish text shows a number of OCR errors, so the relevant check compares the Irish with the English, using
a fuzzy match that takes into account common OCR errors in numbers.

Another tool provides a variation of the common ``find and replace'' operation, but adding translation context:
the replacement will be made if and only if the search patterns for both source and target languages match.
For example, ``dta'' could be a corruption of the Irish ``d\'ata'', or it could be an abbreviation that has been
lowercased. If the English side contains the word ``date'', the replacement can be made with more certainty.

Another tool targets a specific set of errors due to a mismatch in tokenisation: a number of the English sentences
end with a phrase like ``(No. 42)'', whereas, due to poor or missing tokenisation on the Irish side, their 
translations end with ``(Uimh.'', without the number. The tool finds such sentences and appends the number and
right bracket from the English side to the Irish.

Finally, another tool checks if either source or translation is missing, or is identical to the other side, to target
a large number of sentence pairs where the English was placed on both sides of the translation. Although, due to
the nature of the corpus, there are a number of instances where the sentences contained amounts in euros related to fines, and are therefore
legitimately identical on both sides, omitting these number-only translations is not likely to have much of a negative
impact on an SMT system trained on such data; the same cannot be said about including hundreds of sentences of English
in place of Irish.

One other check that was attempted, but ultimately discarded, was influenced by the presence of the names of
translators in the KDE4 corpus. I attempted to train a statistical Named Entity Recognition (NER) system, with the
intent of checking both sides of the corpus to ensure both sides contained equivalent entities, and to print
a warning if not. Due to the nature of the training data, the NER system is quite successful at recognising
Irish names (i.e., containing a patronymic particle), but quite poor otherwise. This meant that the matching
of the Irish NER model was much less capable than the pretrained English model, so the experiment was
ultimately abandoned as it was unable to work as intended.


\begin{figure*}
\begin{center}
\begin{tabular}{l|r|r|r}
\hline
Corpus & \# lines & \# English words & \# Irish words \\
\hline
DGT & 36,275 & 864,373 & 950,500 \\
EU Bookshop & 121,042 & 2,606,607 & 2,704,091 \\
EU constitution & 6,267 & 125,553 & 126,355 \\
Focale & 213,683 & 414,730 & 440,228 \\
GNOME & 75,051 & 288,916 & 297,882 \\
Irish legislation & 132,314 & 2,691,928 & 2,792,595 \\
KDE4 & 110,138 & 439,273 & 523,614 \\
News-2007 (English) & 3,782,548 & 90,490,396 & / \\
Ubuntu & 191 & 1,038 & 1,103 \\
Wikipedia Titles & 17,421 & 35,165 & 36,760 \\
Irish sent. bank & 3,895 & 31,655 & 32,800 \\
Food and Beverages & 339 & 696 & 712 \\
Wikipedia (Irish) & 246,290 & / & 4,047,229 \\
Textbooks & 373,401 & 5,929,635 & 6,568,295 \\
Apertium & 720 & 804 & 791 \\
\hline
total (parallel) & 1,096,117 & 13,573,234 & 14,684,356 \\
\hline
\end{tabular}
\end{center}
\caption{Summary of available corpora for English and Irish, reproduced from \protect\citet[p. 568]{ARCAN16.9}}
\label{fig:smtcorpora}
\end{figure*}

%OpenNLP namefinder - person
%       TOTAL: precision:   91.79%;  recall:   86.82%; F1:   89.23%.
%      person: precision:   91.79%;  recall:   86.82%; F1:   89.23%. [target: 14631; tp: 12702; fp: 1136]


%\chapter{Citizen's Information Corpus}
\label{app:citinfcorpus}

Citizen's Information, available at \href{http://www.citizensinformation.ie}{http://www.citizensinformation.ie}, 
is an Irish government website run by the Citizens Information Board, a national agency responsible for
providing information about government services. 

As a source of parallel text, it has a number of desirable qualities: first and foremost, a large amount of its
content has been translated from English to Irish, while a smaller number of documents are additionally 
available in French, Polish, and Romanian.

Where translations are available, they are highlighted in a box on the right hand side of the page; although
this in itself makes finding pairs of parallel texts quite simple, the address design of the site makes it
even more simple: each document is prefixed first by its two-character ISO 639-1 code, followed by categories
and subcategories the document fits into, followed finally by the page title. Translated documents retain the
English address, changing only the ISO 639-1 language prefix.

Where a translation has been removed -- as happens when the English original has been updated, but the translation
lags, and would otherwise provide inaccurate information -- the page redirects to the English version. This in
itself would normally make it necessary to run language identification software over the non-English documents, to
verify that the document is in the language that the ISO 639-1 prefix of the address would suggest, but the site
generator makes this unnecessary: the \texttt{<html>} tag, the root level element of all HTML documents, is
generated with a \texttt{lang} attribute that contains the correct ISO 639-1 language code, thus making language 
identification unnecessary.

Boilerplate removal is an important step in producing any kind of corpus from web-based texts. One reason why,
in general, is that the inclusion of frequently repeated text items, such as those appearing in site 
navigation menus, can lead to misleading statistics. For parallel text in particular, the inclusion of a
language-choosing element can itself lead to misleading translations: the spot offering ``B\'earla'' in the
Irish version of a website would quite naturally offer ``Irish'' in the same location of the English version
of the page, leading to an incorrect pairing of ``B\'earla'' and ``Irish''.

The high quality page generation of the Citizen's Information site again provides assistance. On each generated
page, the content of the page is separated clearly from the boilerplate with cues in the HTML markup:

\begin{verbatim}
<!-- start of Document -->
<a name="startcontent"></a>
\end{verbatim}

One of the most attractive qualities of the Citizen's Information website is that it is one of the
first government-run websites to fully embrace the recommendations on the Re-use of Public Sector Information,
with all content of the site available under the recommended Open Content licence, Creative Commons CC-BY 4.0.
The terms of this licence allow content that is covered by it to be re-used, modified, and/or redistributed,
as long as an appropriate acknowledgement of the source of the information is provided. This permissive
licence means that any linguistic resources resulting from the website can be used at will, without relying
on research exemptions, and can therefore be used by commercial entities as well as by other researchers.

Finally, that the content of the Citizens Information site is available as HTML is an advantage. Revenue.ie,
the Irish Tax and Customs website, is also available under an open licence, with a large amount of parallel
text in English and Irish, but the majority of its documents are available as PDF. HTML, as a format designed
for on-screen display, is much easier to align at a document level: for example, paragraphs are marked
semantically, with the decision of how they ought to be rendered left to the web browser -- the site designer
may set a number of properties using style sheets, etc., to control how they are presented, but these
properties can be ignored, which is quite often desirable when the reader is using technological solutions
for accessibility, such as screen readers for the blind.

PDF, on the other hand, is designed for print layout. Although PDF can additionally contain the plain text
that is visible within the document, it is not always the case that it is present; even when present, the
PDF creation software may have used its own internal encoding system for rendering characters, such as vowels
with fadas; also, if the document was not ``digital native'', but created through scanning, the text may have
been added using Optical Character Recognition (\GLS{OCR}), which may introduce errors of its own: a common
example is ``arid'' in place of ``and'', due to the visual similarity of \texttt{ri} and \texttt{n}.

\section{Conversion}

Because of the relatively small amount of processing required, due to the nature of the Citizen's 
Information website, much of the usual processing involved in preparing a bilingual corpus from the web
could be bypassed: documents were paired based on their address, language content checked based on that
reported in the \texttt{<html>} tag, and a simple script (included in the supplemental data repository, and
is also reproduced in Appendix~\ref{app:bpremoval}.

\subsection{Phrase based training}

For the phrase based models, because a comprehensive examination of the translation performance on various
parallel corpora was available, and because of the poor quality of a significant portion of that data,
I decided to compare the existing public results with the result of using the cleaning script that
accompanies the Moses decoder, which removes sentences where the ratio of words is 9:1 or greater. 
As well as that, knowing that the oddity of case folding in Irish would
mean that using the standard lower casing method would lead to the presence of a second, artificial form
of a number of words, I decided to additionally compare the performance of systems trained with and
without correct case folding. The script used to do so is relatively uncomplicated, and is reproduced in 
figure~\ref{fig:tolowersh}. Moses~\citep{Koehn:2007:MOS:1557769.1557821} was used for all SMT experiments.

\begin{figure}
\begin{verbatim}
#!/bin/bash
TRANSLIT=$(cat <<'__HERE__'
$wordBoundary = [ [:Z:] [:P:] ];
::NFD();
$wordBoundary { n } [AEIOU] > n\- ;
$wordBoundary { t } [AEIOU] > t\- ;
::Any-Lower;
::NFC();
__HERE__
)

uconv -x "$TRANSLIT"
\end{verbatim}
\label{fig:tolowersh}
\caption{Script to perform correct lowercasing of Irish.}
\end{figure}

\subsection{Factored models}

The same corpus was used for factored models as phrase-based; the data that was part of speech tagged was the cleaned version,
with the Irish lowercased with the Irish-specific method.

\subsubsection{Processing}

Despite the availability of rule-based tools for the morphological analysis and tagging of Irish, IrishFST~\citep{elain09}, 
I chose to train a set of statistical models for the tagging and lemmatisation of Irish. 
The primary motivation was to use a method where a single tagged wordform was selected, correct or not -- the constraint
grammar used to disambiguate the output of IrishFST makes no guarantee that its output will be unambiguous, which poses
a difficulty in systems that expect a single tagged output form.

A secondary motivation was to follow the workflow most amenable to practitioners of statistical NLP, who tend to prefer not 
to use rule-based tools where possible, albeit with certain exceptions -- for example, the sentence splitting and word tokenisation
tools provided originally by the Moses SMT toolkit, which are widely used by other MT and NLP systems, are quite
decidedly rule-based: the sentence uses a list of known abbreviations to avoid incorrect sentence breaks, much like
the components used in Translation Memory systems, which typically use Segmentation Rule eXchange (\GLS{SRX})~\citep{milkowski09srx}.

The options for the statistical processing of Irish are somewhat limited. The only publicly available
model that I was able to find was for Google's SyntaxNet dependency parser, which was trained on the
data from the Universal Dependencies project~\citep{lynn2016universal}. As this dataset is quite small,
I chose instead to train both a part of speech tagging model and lemmatisation model on the Irish Dependency Treebank~\citep{lynn16treebank},
using Apache  OpenNLP\footnote{\href{https://opennlp.apache.org/}{https://opennlp.apache.org/}}, primarily because of prior
familiarity, although OpenNLP as a toolset is designed with a quite broad-minded approach to NLP: most of
the standard facilities are statistically oriented, but several co-exist with rule-based equivalents.

A number of corrections were applied to the tagging of the corpus, primarily to repair the output of
the guesser module of IrishFST, both to reduce the size of the tagset, and to repair its 
lemmatisation\footnote{The corrected version is available online: \href{https://github.com/jimregan/IrishDependencyTreebank}{https://github.com/jimregan/IrishDependencyTreebank}} -- 
most of the words with guessed tags were compounds, for which the lemma of the head of the compound was
given in place of a lemma for the whole.

\subsubsection{Multiwords}

The Irish Dependency Treebank, having been initially tagged with IrishFST, makes use of its tokenisation, and therefore
contains a number of tokens that represent multiword entities. As these cannot easily be reproduced in
other systems, the individual words that these multiwords are composed of needed to be retagged 
individually.

Splitting the multiwords into their individual words has posed some difficulties: in a number of cases, 
the multiwords contain calcified words which no longer have meaning in modern Irish. For example, 
``moite'' in the phrase ``c\'e is moite'' (\textit{except}) is absent from dictionaries of the 
modern language, except as part of that phrase\footnote{eDIL (\href{http://www.dil.ie/32548}{http://www.dil.ie/32548}) lists it as a comparative of ``m\'or''}, 
and ``dt\'i'' in the phrase ``go dt\'i'' is a calcified present subjunctive form of the verb ``tar'', 
but can legitimately be analysed in the modern language as the eclipsed forms of two nouns which 
share the lemma ``t\'i'': a feminine noun meaning \textit{line} or \textit{track}, and a masculine 
noun meaning \textit{tee} (as in golf). These were retagged along linguistic lines, both because
of the lack of an acceptable alternative, and to not introduce confusion for similar contexts: in
these examples, of a comparative adjective after ``is'', and of a subjunctive following ``go''.

The names of political parties are typical of the opposite problem: for example, ``Sinn F\'ein'' is
composed of two pronouns, yet behaves as a single (masculine) proper noun. In such cases, both
components were marked with the tags of the whole.

Even though it is not a meaningful metric to evaluate a statistical NLP tool on the data it was trained on,
where it can be expected to achieve misleadingly high levels of accuracy, the reported accuracy of 99.4\%
seemed, if anything, somewhat low. OpenNLP's evaluation tool can report on which words failed to meet
expectations, and a quick examination was somewhat informative: although in large part, the errors were 
genuine, such as in the phrase ``as ascaill a mháthar'', where for ``a'', OpenNLP predicted the infinitival 
particle, rather than the masculine possessive determiner (meaning \textit{his}), there were some ``errors''
which were not errors on OpenNLP's account, but rather errors in the training data: for example, in the phrase
``maidir le bhfíorú'', ``le'' in the corpus was mistagged with \texttt{Prep:Poss:3P:Pl} -- which cannot be 
generated by IrishFST -- rather than the correct \texttt{Prep:Simp}, which OpenNLP correctly predicted. 
Similarly, in ``uair an chloig'', the word ``chloig'' in the corpus was mistagged with 
\texttt{Noun:Masc:Gen:Sg:Len}, rather than the correct \texttt{Noun:Masc:Gen:Sg:DefArt}, which OpenNLP also 
correctly predicted.

As OpenNLP has the option to include a tag dictionary, an attempt was made to include the entire output
of IrishFST as a tag dictionary; however, this attempt was unsuccessful because of the extremely high
number of tags (3098) that are present in IrishFST but absent from the training data.




\section{Neural Machine Translation}
\label{sect:ch3nmt}

The data for Neural Machine Translation was processed initially in the same way as the data for factored models,
then tokenised using BPE~\cite{sennrich2015neural}. Training and translation was performed using the OpenNMT~\citep{2017opennmt}
toolkit.
\chapter{Comparison and Evaluation}
\label{chap:compare}
\epigraph{\textit{\small{The thing I don't like about tomato sauce is the tomatoes}}}{``Oliwka N.''}

The original intention of this section was to compare the performance of the Apertium-based system
with an SMT system; however, as the rule conversion remains incomplete, no comparison is possible.

Despite the development of a tool for corpus cleaning, there was little in the two problematic
corpora that could be targetted; to check what effect, if any, cleaning the corpus would have,
I used the script distributed with Moses that removes sentence pairs whose relative length
differs by a ratio of 9:1 or more.

Secondly, to compare the effect of proper lower casing, I compared the cleaned corpus when lowercased
using the default method, and with the method for Irish. The results of the three phrase-based systems
and the Neural MT system are presented in figure~\ref{fig:tokcleancompare}.

For comparison, the results on the various subcorpora listed in the previous chapter are reproduced
from \citet[p. 569]{ARCAN16.9} in figure~\ref{fig:iriseval}.

\begin{figure*}
\begin{center}
\begin{tabular}{|l|l|r|r|r|}
\hline
 &  & \multicolumn{3}{|c|}{English to Irish} \\
\hline
\# & Corpus & BLEU & METEOR & chrF \\
\hline
0 & DGT* & 32.39 & 28.45 & 56.75 \\
1 & +EU Bookshop* & 54.54 & 39.03 & 67.82 \\
2 & +EU constitution* & 53.97 & 38.63 & 67.21 \\
3 & +Focal & 54.82 & 39.30 & 68.59 \\
4 & +GNOME* & 55.62 & 40.11 & 68.76 \\
5 & +Irish legislation & 55.77 & 40.01 & 68.62 \\
6 & +KDE4* & 56.62 & 40.73 & 69.36 \\
7 & +News-2007 (mono. English) & / & / & / \\
8 & +Ubuntu & 54.67 & 39.60 & 68.06 \\
9 & +Wikipedia Titles & 55.44 & 39.99 & 68.10 \\
10 & +Irish sent. bank & 55.76 & 40.23 & 68.35 \\
11 & +Food and Beverages & 54.83 & 39.57 & 67.41 \\
12 & +Wikipedia (mono. Irish) & 54.47 & 39.34 & 66.88 \\
13 & +Textbooks & 55.84 & 40.28 & 68.61 \\
14 & +Apertium & 54.85 & 39.66 & 67.75 \\
\hline
 & Google Translate & 40.07 & 33.23 & 65.93 \\
\hline
\end{tabular}
\end{center}
\caption{Summary of automatic evaluation of available corpora for English and Irish, reproduced from \protect\citet[p. 569]{ARCAN16.9}.
Evaluation data was extracted from the corpora marked with `*'.}
\label{fig:iriseval}
\end{figure*}

\begin{figure*}
\begin{center}
\begin{tabular}{|l|l|r|r|r|}
\hline
\multicolumn{5}{|c|}{English to Irish} \\
\hline
Clean & Lowercasing    & BLEU  & METEOR & chrF \\
\hline
No    & Moses          & 31.01 & 27.83  & 68.04 \\
Yes   & Moses          & 30.20 & 27.47  & 67.29 \\
Yes   & Irish specific & 30.27 & 27.73  & 67.66 \\
\hline
Yes   & Irish specific & 02.02 & 04.34  & 17.04 \\
\hline
\end{tabular}
\end{center}
\caption{Comparison of phrase based model trained on an unclean corpus, lowercased using Moses' default; cleaned and default; cleaned and Irish specific; and OpenNMT.}
\label{fig:tokcleancompare}
\end{figure*}


\chapter{Discussion}
\label{chap:discuss}
\epigraph{\textit{\small{I feel so lost when I don't know what I should be complaining about}}}{``Oliwka N.''}

Although using the set of test sentences from \citet{ARCAN16.9} originally seemed a good idea, as it
is a publicly available test set, from a paper with extremely well described data sources, it was, in
hindsight, a poor decision: the test set was made available, but the tokenisation method was not; the
data sources were well described, but the methods employed in processing that data were not. Consequently,
the comparison is somewhat meaningless. Another problem is that the corpora used as sources of the test
data include the two with the worst data quality issues, EU Bookshop and KDE4.

It is perhaps not surprising, then, that contrary to expectations, there is a slight decrease in the evaluation
scores on the SMT systems that were cleaned, and that were lower cased using a method more suitable for Irish --
the test sentences did not undergo such cleaning, and were lower cased using the standard, unsuitable method.

The most disappointing result was that of the Neural MT system. On inspection, much of the output is severely truncated,
with many sentences containing a single character of output, with 621 of the sentences (out of 2000) containing fewer 
than 10 characters (compared with 108 in the reference translation). This may be due to the problem of 
under-translation~\citep{Tu:2016:ACL}. Due to my occasional involvement in the Tesseract OCR 
project~\footnote{\href{https://github.com/tesseract-ocr/tesseract}{https://github.com/tesseract-ocr/tesseract}}, which recently
adopted a neural based system, I had been aware that people had encountered problems with words, lines, or whole
pages going unrecognised. To test this for myself, I trained a model for Irish \textit{seanchl\'o}, or insular 
script\footnote{The trained model is available from \href{https://github.com/jimregan/tesseract-gle-uncial/releases}{https://github.com/jimregan/tesseract-gle-uncial/releases}},
and observed that there were missing words in each of the handful of test pages I tried, although nowhere near to
the same extent as with Neural MT.

Another disappointment is the failure of the factored model: although a factored model was trained, any attempt to
run a translation using it lead to a segmentation fault. The most commonly cited cause of such problems is unescaped
characters in the corpus, although I found nothing of the kind, and if that were the problem, I would expect the 
regular phrase-based models trained on the same corpora to exhibit the same problem.

% chap6.tex (Significance and Future Work)

\chapter{Conclusion}
\label{chap:conclusion}
\epigraph{\textit{\small{But tomorrow was yesterday and now it's today}}}{``Oliwka N.''}

My aim was to describe the development of an English to Irish rule-based machine translation
system, describing the conversion of the dictionaries to create its lexicon, the conversion
of the morphological analyser of Irish to use as its generation lexicon, the creation of a tool
to more easily write rules for the Apertium platform, and the set of rules used in the English-Irish 
system. I have only been able to deliver on the first two points, due to an unforeseen complication
in the creation of the rule conversion tool, and my lack of foresight to allow for such an
eventuality.

A secondary goal, to compare the RBMT system with a Statistical MT system, cannot then be fulfilled;
however, the unexpected complication of poor training data has lead to the creation of a
prototype tool that is already useful for cleaning a small number of problems in parallel corpora.

All software and data created during the course of this dissertation is publicly available, with
the software all open source\footnote{All software, data, and pretrained models are collected
here: \href{https://github.com/jimregan/dissertation-support/}{https://github.com/jimregan/dissertation-support/}.}.

\section{Ongoing Work}

Work on rule conversion for Apertium is ongoing -- it was put on hold two days before the final 
deadline of this dissertation -- consisting of manually expanding the rules. Although it is a matter of
immense personal disappointment to not have been able to get the rule expander to work, even within
the extended timeslot afforded by the dissertation extension, it has not been without benefit. The
rules that deal with the indirect relative, translating ``whose'', were needlessly complicated: as
currently written, they generate a relative chunk and an NP chunk at chunk time, with an interchunk
rule that selects the correct form of the possessive determiner ``a'' from the preceding NP; this
can be written instead as a chunking rule with a single output chunk, tagged as an indirect 
relative, with the interchunk rule modified to insert the relative chunk. In the same vein, 
the expansion of regular NP rules means that rules written earlier are updated to
add macros that perform operations that were not considered earlier.

\section{Future Work}

Reworking the expander to also work for multi-part rules seems a matter of splitting the expansion
process into separate parts for single- and multi-part rules, though it may be beneficial to
rewrite the expander in a more imperative style of programming, to have more control over the
flow of execution.

Despite the poor quality of the bilingual corpora, there is much there that can be used to augment
the RBMT lexicon and rules. Additionally, there is potential for using WordNet~\citep{oregan2016lemongawn}
as a source of lexical entries.

The usual approach to augmenting Apertium translators with existing data is incremental, 
and integrationist, to ensure that to the maximum degree possible that each new item added 
to the lexicon can be analysed and generated correctly. This is in contrast to the dictionary conversion process
proscribed here, which was more completionist: as a result, there was little opportunity to 
ensure the various lexica were consistent with each other. ``The quality of translation is not
proportional to the completeness of the
description  in  the  dictionary.  A  larger
number  of  equivalents  \textit{[\dots]}  often  results  in
decreasing  rather  \textit{[than]}  increasing  the
standard of translation''~\citep[p. 104]{jassem04pwn}. Despite the issues of data quality, where
the SMT data and dictionaries do agree, there is potential to automate the refining of the lexicon,
to remove some of the less helpful translation equivalents.


During the writing phase of this dissertation, a new component was being developed for Apertium
to handle discontiguous multiword expressions. A number of entries in the lexicon are currently disabled
because they require precisely this kind of module, and as they typically contain a parenthetical
expression that's indicative of the inner component (e.g., ``(duine, rud)'' indicates NP), it should
be relatively uncomplicated to generate rules for this new module from them.

The rules in their unconverted form are based on statistical alignment templates; the work on expressing
the components of an SMT system as a single weighted finite state transducer~\citep{Iglesias:2009:HPT:1620754.1620817}
could potentially be applied to them, with the result combined with a statistical set of rules, for a new
kind of hybrid system. 

Duckegg, despite being a proof-of-concept, has potential to be a useful, general purpose tool for cleaning
parallel corpora. Rather than using hard-coded regular expressions, it would be worth investigating the
use of a language designed for defining such textual patterns, such as Apache UIMA's Ruta~\citep{kluegl2016ruta} language.

I still feel that there is merit to the idea of checking both sides of the corpus with Named Entity Recognition. Although
the data used to train the Irish models had its limitations, there is scope for combining those models with a gazetteer, if
only to expand the amount of annotation in the training data, before retraining.

\citet{SkadinsEA:LREC14} describe using a finite state transducer to remove the extra spacing between words in the EU Bookshop
corpus. The Kyoto Fst Decoder\footnote{\href{http://www.phontron.com/kyfd/}{http://www.phontron.com/kyfd/}}~\citep{neubig09interspeech} 
can be used for this; the transducer could also be adapted to take
into consideration the OCR errors, and words that are missing spaces.



%%%%%%%%%%%%%%%%%%%%%%%%%%%%%%%%%%%%%%%%%%%%%%%%%%%%%%%%%%%%%%%%%%%%%%%%%%%%%%%%
%% Bibliography:
%%
\cleardoublepage
\phantomsection
\addcontentsline{toc}{chapter}{Bibliography}
\bibliography{refs}



%%%%%%%%%%%%%%%%%%%%%%%%%%%%%%%%%%%%%%%%%%%%%%%%%%%%%%%%%%%%%%%%%%%%%%%%%%%%%%%%
%% Appendix:
%%

\appendix

\chapter{Citizen's Information Corpus}
\label{app:citinfcorpus}

Citizen's Information, available at \href{http://www.citizensinformation.ie}{http://www.citizensinformation.ie}, 
is an Irish government website run by the Citizens Information Board, a national agency responsible for
providing information about government services. 

As a source of parallel text, it has a number of desirable qualities: first and foremost, a large amount of its
content has been translated from English to Irish, while a smaller number of documents are additionally 
available in French, Polish, and Romanian.

Where translations are available, they are highlighted in a box on the right hand side of the page; although
this in itself makes finding pairs of parallel texts quite simple, the address design of the site makes it
even more simple: each document is prefixed first by its two-character ISO 639-1 code, followed by categories
and subcategories the document fits into, followed finally by the page title. Translated documents retain the
English address, changing only the ISO 639-1 language prefix.

Where a translation has been removed -- as happens when the English original has been updated, but the translation
lags, and would otherwise provide inaccurate information -- the page redirects to the English version. This in
itself would normally make it necessary to run language identification software over the non-English documents, to
verify that the document is in the language that the ISO 639-1 prefix of the address would suggest, but the site
generator makes this unnecessary: the \texttt{<html>} tag, the root level element of all HTML documents, is
generated with a \texttt{lang} attribute that contains the correct ISO 639-1 language code, thus making language 
identification unnecessary.

Boilerplate removal is an important step in producing any kind of corpus from web-based texts. One reason why,
in general, is that the inclusion of frequently repeated text items, such as those appearing in site 
navigation menus, can lead to misleading statistics. For parallel text in particular, the inclusion of a
language-choosing element can itself lead to misleading translations: the spot offering ``B\'earla'' in the
Irish version of a website would quite naturally offer ``Irish'' in the same location of the English version
of the page, leading to an incorrect pairing of ``B\'earla'' and ``Irish''.

The high quality page generation of the Citizen's Information site again provides assistance. On each generated
page, the content of the page is separated clearly from the boilerplate with cues in the HTML markup:

\begin{verbatim}
<!-- start of Document -->
<a name="startcontent"></a>
\end{verbatim}

One of the most attractive qualities of the Citizen's Information website is that it is one of the
first government-run websites to fully embrace the recommendations on the Re-use of Public Sector Information,
with all content of the site available under the recommended Open Content licence, Creative Commons CC-BY 4.0.
The terms of this licence allow content that is covered by it to be re-used, modified, and/or redistributed,
as long as an appropriate acknowledgement of the source of the information is provided. This permissive
licence means that any linguistic resources resulting from the website can be used at will, without relying
on research exemptions, and can therefore be used by commercial entities as well as by other researchers.

Finally, that the content of the Citizens Information site is available as HTML is an advantage. Revenue.ie,
the Irish Tax and Customs website, is also available under an open licence, with a large amount of parallel
text in English and Irish, but the majority of its documents are available as PDF. HTML, as a format designed
for on-screen display, is much easier to align at a document level: for example, paragraphs are marked
semantically, with the decision of how they ought to be rendered left to the web browser -- the site designer
may set a number of properties using style sheets, etc., to control how they are presented, but these
properties can be ignored, which is quite often desirable when the reader is using technological solutions
for accessibility, such as screen readers for the blind.

PDF, on the other hand, is designed for print layout. Although PDF can additionally contain the plain text
that is visible within the document, it is not always the case that it is present; even when present, the
PDF creation software may have used its own internal encoding system for rendering characters, such as vowels
with fadas; also, if the document was not ``digital native'', but created through scanning, the text may have
been added using Optical Character Recognition (\GLS{OCR}), which may introduce errors of its own: a common
example is ``arid'' in place of ``and'', due to the visual similarity of \texttt{ri} and \texttt{n}.

\section{Conversion}

Because of the relatively small amount of processing required, due to the nature of the Citizen's 
Information website, much of the usual processing involved in preparing a bilingual corpus from the web
could be bypassed: documents were paired based on their address, language content checked based on that
reported in the \texttt{<html>} tag, and a simple script (included in the supplemental data repository, and
is also reproduced in Appendix~\ref{app:bpremoval}.
\chapter{Citizen's Information Boilerplace Removal}
\label{app:bpremoval}

Because of the cues provided by the generated HTML of the Citizen's Information website, only a trivial
script was required to remove boilerplace, based on a pair of comments: one containing the phrase
``start of Document'', the other beginning with ``Generated:'', and followed by the date and time the
current document was last updated.

\begin{verbatim}
#!/usr/bin/perl

use warnings;
use strict;
use utf8;

binmode(STDIN, ":utf8");
binmode(STDOUT, ":utf8");

my $reading = 1;
while(<>) {
    if (/<!-- start of Document -->/) {
        $reading = 1;
        next;
    }
    if (/<!-- Generated:/) {
        $reading = 0;
        next;
    }
    if(m!</head>!) {
        $reading = 0;
        print;
        next;
    }
    next if(/<div id="lastupdated"/);
    next if(/<meta name="google-site-verification"/);
    next if(/<link/);
    next if(/<script/);
    if($reading) {
        print;
    }
}

END {
    print "</html>\n";
}
\end{verbatim}





%%%%%%%%%%%%%%%%%%%%%%%%%%%%%%%%%%%%%%%%%%%%%%%%%%%%%%%%%%%%%%%%%%%%%%%%%%%%%%%%
%% Index:
%%
\printthesisindex

\end{document}
