%\def\longs{{\fontencoding{TS1}\selectfont s}}
\chapter{Introduction}

% The TODO for this chapter
This introduction will provide a high level overview of Machine Translation (\GLS{MT}) (section~\ref{sect:intromt}), and of the 
particular challenges posed by Irish (section~\ref{sect:introirish}).
It will also describe briefly the approach to evaluation 
which will be employed (section~\ref{sect:introeval}), and introduce the structure of the remainder of the dissertation.



\section{Machine Translation}
\label{sect:intromt}

Machine Translation (MT), or automatic translation, is an automated process that recodes one human language 
into another. Several approaches exist to tackling the problem, which can be classified in a number of 
ways: of those considered in this dissertation, the primary distinction is between knowledge-based (linguistic) approaches, 
using dictionaries and grammatical rules to perform the translation; and data-driven approaches, which 
learn to translate using a corpus of existing translations.

Machine Translation is used for two primary purposes: \textit{assimilation}, to get the gist of text in 
a foreign language, and \textit{dissemination}, as an input to publication, typically post-edited by 
translators. Free online services, such as Google Translate, are usually intended for assimilation, 
while the translation services in use at multilingual organisations such as the European Union are 
typically intended for dissemination. Consequently, systems can be designed with certain 
trade-offs in mind: for example, a system intended primarily for assimilation may trade accuracy for 
broader coverage, and vice versa.

The amount of data and effort required to build an MT system for either task depends on a number of 
factors, first among which is how closely-related the languages are. More closely-related languages 
tend to share similar structures, and tend to have more cognate words (which, even when unrecognised 
by the translator, pass through as ``free rides''). \cite{trosterud2012evaluating} compare a gisting 
system from North S\'ami to Norwegian Bokm{\aa}l (non-Indo-European to Indo-European) with their earlier 
work on Norwegian Nynorsk and Bokm{\aa}l~\citep{unhammer2009rfr}, which achieves near post-edition quality 
using less than a third of the amount of transfer code. 

Machine Translation, ultimately, is a tool for facilitating communication, and the needs of communication 
tend to dictate the purpose: when translating from a major language to a minor language, assimilation 
is the primary need of the majority of (potential) users, whereas when translating from a major language 
to a minor language, the primary need is dissemination, to gain more text in the minor language.

\subsection{Technology and challenges}

Knowledge-based, or Rule-Based Machine Translation (\GLS{RBMT}), systems can be classified into three main categories: \textit{direct translation}, where no 
transformation of the source text is performed; \textit{transfer-based}, where the input is transformed 
into a language-dependent intermediate representation -- typically, \textit{lemma}: the citation 
form of the word; \textit{morphological analysis}: details such as case, number, person, etc.; and 
possibly \textit{semantic analysis}: the subject of a verb, etc. -- and \textit{interlingua}, where the 
input is transformed into a language independent representation. 

Rule-based systems tend to be costly and time-consuming to build, as they require lexicographers and 
linguists to create dictionaries and rules. Transfer-based and interlingua approaches, as they operate 
on abstract representations, generalise better -- if a word is in the dictionary, it can be handled in 
all forms -- but they tend to handle exceptions, such as idioms, poorly: specific rules must be written, 
and multiple rules may conflict with each other. Additionally, the complexity of an interlingua tends,
in reality, to increase with the addition of each additional language, as each new language adds its
unique terms and structures, the existing lexical and syntactic rule base needs to expand to 
accommodate them.

The prerequisite for building data-driven systems, including Statistical Machine Translation (\GLS{SMT}),
Neural Machine Translation (\GLS{NMT}), and Example-Based Machine Translation (\GLS{EBMT}), is the 
existence of human-translated bilingual (or multilingual) corpora.

A common source of professionally translated multilingual corpora are international organisations such 
as the United Nations or the European Union, which generate a substantial amount of freely available, 
high-quality multilingual documents (in 24 languages for the EU\footnote{Art. 1 of the Regulation No. 1 
of 15 April 1958 determining the languages to be used by the European Economic Community.} and in 6 
languages for the UN\footnote{Rule 51 of the Rules of Procedure of the General Assembly of the United 
Nations; rule 41 of the Provisional Rules of Procedure of the United Nations Security Council.}).

Compared to rule-based MT systems, data-driven MT systems are cheaper -- at least, for widely-spoken 
languages -- and more flexible: a data-driven system is not designed specifically for one language 
pair, but can accommodate any language pair for which a corpus is available. Also, because data-driven 
MT systems are based on human-translated texts, the output of such systems is, or, at least, can be, 
more natural (or ``fluent''), and it naturally adapts well to exceptions: if the corpus contains the 
phrase, it is effectively not an exception.

\textit{Zipf's law} states that in a given corpus, the frequency of a word is inversely proportional 
to its frequency rank: the most frequent word will occur (approximately) twice as often as the second, 
three times as often as the third, and so on. Conversely, the majority of words (40-60\%) are 
\textit{hapax legomena} (words which only occur once). As data-driven MT is corpus-based, it therefore 
suffers from the problem of \textit{data sparsity} due to the high proportion of hapax legomena: longer 
phrase matches are absent from the translation model; contextual information is absent from the language 
model, affecting the quality (``fluency'') of the output.

Data sparsity is the biggest problem in data-driven MT. Although there have been attempts to solve it 
by using linguistic information, dating back to CANDIDE (the first SMT system), the most common approach 
is to simply add more data\footnote{On the other hand, it has been claimed that translation quality can be 
increased by simply discarding infrequent phrases: \citet{Johnson07improvingtranslation}.}. A large 
amount of websites are available in multiple languages, so crawling the web for parallel text is a 
common method of collecting corpora~\citep{zora80038}, particularly for the providers of free online 
MT, such as Google and Microsoft, who also operate search engines and therefore already have access to 
such data. The use of such data, however, has its own problems, as such documents are often not just 
translated, but \textit{localised}: different units of measurement, currency, and even country 
names~\citep{quince08}, because of their collocation, become ``translations''. 

Crowdsourcing, where online communities are solicited for content, is also in increasing use. Amazon's 
Mechanical Turk, an online market place for work, provides an easy way to pay people small amounts of 
money for small units of work, which has been used in MT~\citep{zaidan2011}. Google has made use of 
crowdsourcing since the earliest days of Google Translate, by providing users with a means of improving 
translation suggestions~\citep{chin07suggest}, by providing a translation memory 
system~\citep{galvez09toolkit}, and more directly, with Translate Community~\citep{kelman14community}. 
Crowdsourcing has been recently used for Irish-English MT, via user contributed translations of 
tweets~\citep{dowling2017crowd}.

Finally, the quality of MT output depends on the quality of the input. Even the most banal imperfections, 
such as misspellings or grammar mistakes, even if they are barely noticeable to a human translator, can 
compromise the most elaborate MT systems~\citep{porsiel12sec}.

\section{Irish}
\label{sect:introirish}

Irish is an Indo-European language, from the Q-Celtic branch of Celtic languages, closely related to 
Scots Gaelic and Manx, and more distantly related to Welsh, Breton, and Cornish. Irish exists in a 
somewhat paradoxical state in that, while it is ``definitely endangered''~\citep{moseley2010atlas}, it
is also the first national language of the Republic of Ireland, and is an official language of the
European Union; although Irish is a compulsory subject for all students\footnote{A limited set of
exemptions are outlined in Circular letter M10/94: 
\href{https://www.education.ie/en/Circulars-and-Forms/Active-Circulars/ppc10_94.pdf}{https://www.education.ie/en/Circulars-and-Forms/Active-Circulars/ppc10\_94.pdf}.},
30.1\% of 10-19 year olds interviewed for the 2016 census answered ``no'' to the question ``can you
speak Irish?''\footnote{\href{http://www.cso.ie/en/media/csoie/releasespublications/documents/population/2017/7._The_Irish_language.pdf}{http://www.cso.ie/en/media/csoie/releasespublications/documents/population/2017/ 7.\_The\_Irish\_language.pdf}};
and despite the existence of designated Irish speaking, or \textit{Gaeltacht}, areas, the same census
reports that only 32\% of the Irish speakers living in Gaeltacht areas speak Irish on a daily basis.

There are three primary dialects of Irish, corresponding to three of the four provinces of Ireland:
Connacht, Munster, and Ulster (there is evidence of a Leinster dialect having existed); there is
a quite low level of mutual intelligibility between the dialects, though this is reducing thanks
to the availability of television and radio in Irish. For the most part, the differences are most
evident in the spoken language, although differences in syntax do exist. A written standard for
Irish has existed since the 1950s, though as it is intended solely to provide a written standard
for official publications, and ``not an official standard for speech nor a denial of the linguistic wealth of the various dialects''
(``Ní caighdeán oifigiúil le haghaidh na cainte atá i gceist ná ní séanadh ar an saibhreas teanga sna canúintí éagsúla é.''~\citep[xviii]{caighdean2017}).

Unlike most other European languages outside of the Celtic family, Irish is a VSO language,
with adjectives and other modifiers typically following the noun or noun phrase.
Irish is an inflected language, with suffix or internal inflection, as well as a set of initial
mutations which are typical of Celtic languages. Nouns have gender -- masculine or feminine --
and inflect for case and number. Nouns can have three cases: a common nominative/accusative,
genitive, and vocative. There are also calcified remnants of a dative case, which is still
in use in some areas.

Verbs inflect for tense and mood, and may, particularly in the Munster dialect, also inflect for person
and number. Verbs are quite
regular, although a small number of the more common verbs are extremely irregular. Irish has
two equivalents of the verb ``to be'': the existential verb ``b\'i'' and the copula ``is'', 
which share the same Indo-European origins as the English ``be'' and ``is'', or Polish 
``by\'c'' and ``jest''. Irish has no single equivalents of ``yes'' or ``no'', using instead
the repeated verb (or copula) or its negation to express an affirmative or negative response.

\subsection{Problems facing Irish}

\subsubsection{Lack of linguistic research}
\label{sssect:lingresearch}

``In the area of syntax, basic linguistic research on the linguistic phenomena of Irish syntax is 
required''~\citep[p. 73]{LWP-Irish}. One such example is reflexive verbs and verb phrases in Irish.
While some study exists~\citep[e.g.,][]{nolan2004rrg,nolan2000reflexive}, they focus mainly on the use of the
reflexive pronoun, ``f\'ein'', while there are other constructs which show at least basic reflexivity, such
as verbs which express their object with an adjunct: ``chuir m\'e mo ch\'ota orm'' (\textit{I put my coat on 
(myself)}, ``Brostaigh ort!'' (\textit{Hurry (yourself) up}).

Similarly, the infinitive of a verb is
expressed with a verbal noun preceded by the infinitival particle ``a''; however, in constructs where
the object of a verb or copula phrase is an infinitival phrase, the infinitival particle is dropped
where there is no object of the infinitival verb: e.g., ``is maith liom leabhar a l\'eamh'' 
(\textit{I like to read a book}) retains the infinitival particle, but 
``is maith liom l\'eamh'' (\textit{I like to read}) does not, but grammars of Irish do not typically
speak of this as being the same process; for example, \citet[p. 128]{grammar1980cb} explicitly mention
in point 7 that the negative infinitive is expressed with the preposition ``gan'' and the verbal noun,
while indirectly mentions in point 9 that the equivalent when there is an object uses the particle ``a'',
but without linking the two.

\subsubsection{Technological support}

\citet[s. 5.1]{krauwer2003basic} defines ``the  Basic  Language  Resource  Kit  
(abbreviated  BLARK)  as  the  minimal  set  of  language  
resources that is necessary to do any precompetitive 
research and education at all'', and includes in it linguistic resources such as dictionaries, 
grammars, and corpora, as well as tools such as text to speech, morphological analysers, taggers,
parsers, and speech recognisers.

Although Irish is considered to have only ``fragmentary support'' in the area of speech processing,
and ``weak/no support'' in the areas of machine translation, text analysis, and speech and text
resources~\citep[pp. 77--78]{LWP-Irish}, there is a perhaps surprising range of technological support,
given the relatively low number of Irish speakers.

For basic text processing purposes, support is available for both spell 
checking, through GaelSpell\footnote{\href{https://borel.slu.edu/ispell/}{https://borel.slu.edu/ispell/}}, and 
grammar checking, through \textit{An Gramad\'oir}\footnote{\href{http://borel.slu.edu/gramadoir/}{http://borel.slu.edu/gramadoir/}}\footnote{Another
grammar checker, based on the ``LanguageTool'' language checker described by \citet{naber2003rule}, was
commenced during the early stages of this work, coupling the text analysis resources of
\citet{elain09} with the grammar checking with the rules of \textit{An Gramad\'oir}, but needs
some additional work to be fully integrated.}.

For text analysis, there is a morphological analyser and disambiguator~\citep{elain09},
a statistical part-of-speech tagger for tweets~\citep{lynn2015minority}, 
a rule-based partial dependency parser~\citep{UDHONNCHADHA10.824}, and a
treebank for statistical dependency parsing~\citep{lynn16treebank}, a subset of which
is available mapped to the Universal Dependencies scheme~\citep{lynn2016universal}.

Speech synthesis (text to speech) for all three dialects is available 
online\footnote{\href{http://www.abair.ie}{http://www.abair.ie}}~\citep{chasaide2017abair}, 
which is employed in a prototype chatbot for Computer Assisted Language Learning
(CALL)~\citep{nichiarain2016chatbot}.

In terms of resources, Irish fulfils many of the requirements of a BLARK. Although there
are relatively few dictionaries, those that are available are available in digital form:
``An Focl\'oir Beag''~\citep{odonaill1991focloirbeag}, the only monolingual dictionary
of modern Irish; ``Focl{\'o}ir Gaeilge-B{\'e}arla''~\citep{donaill1977focloir}, the Irish--English
dictionary of reference; and ``English--Irish Dictionary''~\citep{debhaldraithe1959english}, the 
previous English--Irish dictionary of reference have all been digitised and made available online\footnote{http://www.teanglann.ie/ga/},
while the current English--Irish Dictionary of reference is digital native\footnote{http://www.focloir.ie/}; 
in addition to dictionaries, there is a publicly accessible terminology database\footnote{\href{http://www.tearma.ie/}{http://www.tearma.ie/}}~\citep{mvechura2010focal}, and
as well as human-targeted dictionaries, there are lexicons intended for use in computing, including a WordNet for Irish~\citep{oregan2016lemongawn}
and a high-quality morphological database~\citep{mchura2014}.
In terms of corpora, the New Corpus for Ireland~\citep{Kilgarriff2006EfficientCD} contains 30 million words.

\subsubsection{Character encoding}

Although not a problem peculiar to Irish, the issue of character encoding, or, rather, of incorrectly recoded
text, arises unusually often with Irish texts.

In earlier days of computing, a number of 8-bit encoding schemes were employed in various areas around the world,
to allow the local alphabets to be encoded in text. ASCII, the basic set of characters that are sufficient to
encode the English alphabet along with a number of common characters, such as punctuation and mathematical operators,
consists of less than 128 characters, which can fit into 7 bits of computer memory -- a ``bit'' encodes a single
boolean 1 or 0, so a sequence of 7 holds $2^7$, or 128\footnote{To be clear, 0 to 127}, possibilities. As the computers of the time were typically
able to address 8 bits (one byte) of memory at a time, this left one extra bit, allowing a further 128 characters to be used,
which were typically arranged among the dominant languages of neighbouring countries: Latin1 served the Latin-script
based languages of Western Europe, including the accented vowels used by Irish.

As computers became more powerful, and able to address larger amounts of memory, a coding scheme named Unicode was
with the goal of serving all of the world's languages. Because each character took more than 8 bits to encode,
a number of standardised methods for serialising these characters were devised: the most common of which, UTF-8,
is a \textit{variable-width encoding}, meaning that the higher the number representing the character was, the
more individual bytes were required to represent it. One particular advantage of UTF-8 is that the set of ASCII
characters remain unchanged, with the eighth bit serving to represent that the preceding byte was part of the
same Unicode character. All of the accented vowels used in modern Irish require two bytes, while the remainder
require one. When the characters are incorrectly recoded -- that is, when characters in, for example, UTF-8 are
misinterpreted as having been encoded in Latin1 -- these two bytes are displayed as two individual incorrect
characters: for example, ``{\'e}'' will be recoded as \texttt{é}.


\subsection{Particular challenges of Irish}

While most languages have aspects that can be challenging for particular types of technology, there
are a number of aspects of Irish that are particularly challenging for all language-related processing.

\subsubsection{Numbers}

Numbers in Irish are more complicated than in English; cardinal numbers have a slightly different form
for use when counting, and numbers distinguish between human and non-humans in agreement for numbers
between two and ten, along with twelve.

Numbers in Irish can trigger a mutation in the noun that follows; for example, ``dh\'a'' (\textit{two})
triggers lenition, while ``seacht'' (\textit{seven}) triggers eclipsis. This has posed a problem in
the past for software localisation in systems that were designed with Western European languages in
mind, that could only distinguish between singular and plural, if at all. Irish is far from the most
demanding in its numeric needs -- Slovenian distinguishes between singular, dual, and plural, as well
as the common Slavic genitive of quantity (five or more) -- but has benefited from the needs of other
languages.


\subsubsection{Case folding}
\label{ssect:casefold}

Case folding, in scripts such as the Latin script used by both Irish and English that distinguish 
between upper, lower, and/or title case forms of letters, ``is the process of making two texts identical 
which differ in case but are otherwise `the same'''\citep[s. 2.1]{Phillips:16:CMW}. The conversion from 
upper or title case to lower case is typically referred to as ``lowercasing'' or ``decapitalisation''); 
conversely, the conversion from lower case to uppercase as ``uppercasing'' or ``capitalisation''.

Case folding is important in software that needs to compare text, as the number of combinations in the 
simplest case of a single upper case and single lower case letter is $n^2$, where $n$ is the length of 
the word. For example, the  word ``dog'' has three letters, and yields $3^2 = 8$ combinations: dog, dOg, 
dOG, DOG, Dog, DOg, doG, and DoG.

In English, it's a quite straightforward process to map from upper to lower case by mapping the letters 
from A to Z to the letters from a to z. In other languages, the process is less straightforward: for 
example, the Greek letter sigma ($\Sigma$) has two lower case forms: $\varsigma$ is used only when in 
word final position, while $\sigma$ is used otherwise\footnote{The same was formerly true of English, 
with `s' appearing only in word final position, with the ``medial s'', ſ, used otherwise}. In Dutch, the 
digraph `ij' is considered a single letter ``and thus if a word starts with ij and has to be capitalized, 
both the \textbf{i} and \textbf{j} are affected'', such as in ``IJsland'' 
(\textit{Iceland})~\citep[p. 14]{donaldson2008dutch}.

In Irish, initial mutations (aside from lenition) precede the word proper, and are considered separate
from it: the eclipsed form of the word ``geata'' (\textit{gate}) is ``ngeata'', which is capitalised not
as ``NGEATA'', but as ``nGEATA''. The affixation of a case-invariant part is not unusual: the Zulu 
language uses such prefixes to ``denote proper names endowed with morphological structure as is customary 
in Zulu''~\citep[p. 143]{bosch2011towards}, such as ``isiZulu'' (\textit{the Zulu language}), where the 
case invariant prefix ``isi-'' typically denotes that what follows is a language. It's not uncommon to
see even in English, where a case invariant ``e'' can be used to denote that something is ``electronic'',
or, in possibly the most common instance in current times, in the brand names of Apple's consumer
electronic devices: the iPhone, iPad, and iPod.

The challenge with Irish's prefixes extends with the pre-vocalic mutations, \textit{n}, \textit{t}, and
\textit{h}: \textit{n} and \textit{t} are written in lower case form followed by a hyphen, but without 
when preceding an upper case vowel: ``\'ar n-athair'' (\textit{our father}) is in title case 
``\'Ar nAthair''. Far more complicated is \textit{h}; \textit{n} and \textit{t} have always been part of
the conventional Irish alphabet, and so have always been written in a way that their use as a mutation is
distinct; \textit{h} had traditionally only been used as a prefix, or as a means of marking lenition of the
previous consonant, but modern Irish has imported many words, particularly from Greek, that legitimately
begin with \textit{h}, so to algorithmically convert words beginning with ``h'' to upper or title case 
is not possible without the use of a dictionary\footnote{The closely related Scots Gaelic suffers less
from these problems, as the ``h'' prefix is always written with a hyphen, and the hyphen is retained in
upper case for pre-vocalic mutations.}.

Fortunately, in almost all computing applications, lower case is the norm, which leaves only the 
relatively minor issue of inserting a hyphen between \textit{n} and \textit{t} when their use is as a
mutation.



\section{Evaluation}
\label{sect:introeval}

There are many automatic evaluation metrics for estimating translation quality. All are based on
matching n-grams against a reference translation, though the details vary greatly among the metrics.
As \citet{ARCAN16.9} provide a comprehensive set of evaluations on the range of parallel corpora
available for the English--Irish translation pair, to be able to use their results as a benchmark, I
have opted to use the same evaluation metrics.

BLEU~\citep{Papineni02bleu}, one of the first automatic metrics to be considered comparable with
human judgement, is the de facto standard metric employed when evaluating statistical machine translation
systems. BLEU scores candidate translations against a reference translation, or set of reference
translations, calculating the overlap between the candidate and the reference. The scores for each
segment, or sentence, are averaged across the whole test set, to give a single score.
BLEU's score, although given as a number between 0 and 1 (low to high), is typically reported as a number
between 0 and 100.

METEOR~\citep{denkowski2014meteor} is an automatic metric for translation evaluation that is based on the
harmonic mean of the precision and recall of unigrams, with recall given a higher weight than precision.
Although it has language-specific features for approximate machine, such as stemming and synonyms, these
are supported for a small number of languages (Irish is not included).

chrF~\citep{popovic2015chrf} is a metric for translation evaluation that is based on characters, instead of
words (or, rather, tokens), which is resistant to tokenisation differences, and to slight morphological
errors.

%\section{}